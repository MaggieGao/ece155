
\documentclass[letterpaper,hide notes,xcolor={table,svgnames},pdftex]{beamer}
\def\showexamples{t}


%\usepackage[svgnames]{xcolor}

%% Demo talk
%\documentclass[letterpaper,notes=show]{beamer}

\usecolortheme{crane}
\setbeamertemplate{navigation symbols}{}

\usetheme{MyPittsburgh}
%\usetheme{Frankfurt}

%\usepackage{tipa}

\usepackage{hyperref}
\usepackage{graphicx,xspace}
\usepackage[normalem]{ulem}

\newcommand\SF[1]{$\bigstar$\footnote{SF: #1}}



\newcounter{tmpnumSlide}
\newcounter{tmpnumNote}

% old question code
%\newcommand\question[1]{{$\bigstar$ \small \onlySlide{2}{#1}}}
% \newcommand\nquestion[1]{\ifdefined \presentationonly \textcircled{?} \fi \note{\par{\Large \textbf{?}} #1}}
% \newcommand\nanswer[1]{\note{\par{\Large \textbf{A}} #1}}


 \newcommand\mnote[1]{%
   \addtocounter{tmpnumSlide}{1}
   \ifdefined\showcues {~\tiny\fbox{\arabic{tmpnumSlide}}}\fi
   \note{\setlength{\parskip}{1ex}\addtocounter{tmpnumNote}{1}\textbf{\Large \arabic{tmpnumNote}:} {#1\par}}}

\newcommand\mmnote[1]{\note{\setlength{\parskip}{1ex}#1\par}}

%\newcommand\mnote[2][]{\ifdefined\handoutwithnotes {~\tiny\fbox{#1}}\fi
% \note{\setlength{\parskip}{1ex}\textbf{\Large #1:} #2\par}}

%\newcommand\mnote[2][]{{\tiny\fbox{#1}} \note{\setlength{\parskip}{1ex}\textbf{\Large #1:} #2\par}}

\newcommand\mquestion[2]{{~\color{red}\fbox{?}}\note{\setlength{\parskip}{1ex}\par{\Large \textbf{?}} #1} \note{\setlength{\parskip}{1ex}\par{\Large \textbf{A}} #2\par}\ifdefined \presentationonly \pause \fi}

\newcommand\blackboard[1]{%
\ifdefined   \showblackboard
  {#1}
  \else {\begin{center} \fbox{\colorbox{blue!30}{%
         \begin{minipage}{.95\linewidth}%
           \hspace{\stretch{1}} Some space intentionally left blank; done at the blackboard.%
         \end{minipage}}}\end{center}}%
         \fi%
}



%\newcommand\q{\tikz \node[thick,color=black,shape=circle]{?};}
%\newcommand\q{\ifdefined \presentationonly \textcircled{?} \fi}

\usepackage{listings}
\lstset{%
  keywordstyle=\bfseries,
  aboveskip=15pt,
  belowskip=15pt,
  captionpos=b,
  identifierstyle=\ttfamily,
  escapeinside={(*@}{@*)},
  stringstyle=\ttfamiliy,
  frame=lines,
  numbers=left, basicstyle=\scriptsize, numberstyle=\tiny, stepnumber=0, numbersep=2pt}

\usepackage{siunitx}
\newcommand\sius[1]{\num[group-separator = {,}]{#1}\si{\micro\second}}
\newcommand\sims[1]{\num[group-separator = {,}]{#1}\si{\milli\second}}
\newcommand\sins[1]{\num[group-separator = {,}]{#1}\si{\nano\second}}
\sisetup{group-separator = {,}, group-digits = true}

%% -------------------- tikz --------------------
\usepackage{tikz}
\usetikzlibrary{positioning}
\usetikzlibrary{arrows,backgrounds,automata,decorations.shapes,decorations.pathmorphing,decorations.markings,decorations.text}

\tikzstyle{place}=[circle,draw=blue!50,fill=blue!20,thick, inner sep=0pt,minimum size=6mm]
\tikzstyle{transition}=[rectangle,draw=black!50,fill=black!20,thick, inner sep=0pt,minimum size=4mm]

\tikzstyle{block}=[rectangle,draw=black, thick, inner sep=5pt]
\tikzstyle{bullet}=[circle,draw=black, fill=black, thin, inner sep=2pt]

\tikzstyle{pre}=[<-,shorten <=1pt,>=stealth',semithick]
\tikzstyle{post}=[->,shorten >=1pt,>=stealth',semithick]
\tikzstyle{bi}=[<->,shorten >=1pt,shorten <=1pt, >=stealth',semithick]

\tikzstyle{mut}=[-,>=stealth',semithick]

\tikzstyle{treereset}=[dashed,->, shorten >=1pt,>=stealth',thin]

\usepackage{ifmtarg}
\usepackage{xifthen}
\makeatletter
% new counter to now which frame it is within the sequence
\newcounter{multiframecounter}
% initialize buffer for previously used frame title
\gdef\lastframetitle{\textit{undefined}}
% new environment for a multi-frame
\newenvironment{multiframe}[1][]{%
\ifthenelse{\isempty{#1}}{%
% if no frame title was set via optional parameter,
% only increase sequence counter by 1
\addtocounter{multiframecounter}{1}%
}{%
% new frame title has been provided, thus
% reset sequence counter to 1 and buffer frame title for later use
\setcounter{multiframecounter}{1}%
\gdef\lastframetitle{#1}%
}%
% start conventional frame environment and
% automatically set frame title followed by sequence counter
\begin{frame}%
\frametitle{\lastframetitle~{\normalfont(\arabic{multiframecounter})}}%
}{%
\end{frame}%
}
\makeatother

\makeatletter
\newdimen\tu@tmpa%
\newdimen\ydiffl%
\newdimen\xdiffl%
\newcommand\ydiff[2]{%
    \coordinate (tmpnamea) at (#1);%
    \coordinate (tmpnameb) at (#2);%
    \pgfextracty{\tu@tmpa}{\pgfpointanchor{tmpnamea}{center}}%
    \pgfextracty{\ydiffl}{\pgfpointanchor{tmpnameb}{center}}%
    \advance\ydiffl by -\tu@tmpa%
}
\newcommand\xdiff[2]{%
    \coordinate (tmpnamea) at (#1);%
    \coordinate (tmpnameb) at (#2);%
    \pgfextractx{\tu@tmpa}{\pgfpointanchor{tmpnamea}{center}}%
    \pgfextractx{\xdiffl}{\pgfpointanchor{tmpnameb}{center}}%
    \advance\xdiffl by -\tu@tmpa%
}
\makeatother
\newcommand{\copyrightbox}[3][r]{%
\begin{tikzpicture}%
\node[inner sep=0pt,minimum size=2em](ciimage){#2};
\usefont{OT1}{phv}{n}{n}\fontsize{4}{4}\selectfont
\ydiff{ciimage.south}{ciimage.north}
\xdiff{ciimage.west}{ciimage.east}
\ifthenelse{\equal{#1}{r}}{%
\node[inner sep=0pt,right=1ex of ciimage.south east,anchor=north west,rotate=90]%
{\raggedleft\color{black!50}\parbox{\the\ydiffl}{\raggedright{}#3}};%
}{%
\ifthenelse{\equal{#1}{l}}{%
\node[inner sep=0pt,right=1ex of ciimage.south west,anchor=south west,rotate=90]%
{\raggedleft\color{black!50}\parbox{\the\ydiffl}{\raggedright{}#3}};%
}{%
\node[inner sep=0pt,below=1ex of ciimage.south west,anchor=north west]%
{\raggedleft\color{black!50}\parbox{\the\xdiffl}{\raggedright{}#3}};%
}
}
\end{tikzpicture}
}


%% --------------------

%\usepackage[excludeor]{everyhook}
%\PushPreHook{par}{\setbox0=\lastbox\llap{MUH}}\box0}

%\vspace*{\stretch{1}

%\setbox0=\lastbox \llap{\textbullet\enskip}\box0}

\setlength{\parskip}{\fill}

\newcommand\noskips{\setlength{\parskip}{1ex}}
\newcommand\doskips{\setlength{\parskip}{\fill}}

\newcommand\xx{\par\vspace*{\stretch{1}}\par}
\newcommand\xxs{\par\vspace*{2ex}\par}
\newcommand\tuple[1]{\langle #1 \rangle}
\newcommand\code[1]{{\sf \footnotesize #1}}
\newcommand\ex[1]{\uline{Example:} \ifdefined \presentationonly \pause \fi
  \ifdefined\showexamples#1\xspace\else{\uline{\hspace*{2cm}}}\fi}

\newcommand\ceil[1]{\lceil #1 \rceil}


\AtBeginSection[]
{
   \begin{frame}
       \frametitle{Outline}
       \tableofcontents[currentsection]
   \end{frame}
}



\pgfdeclarelayer{edgelayer}
\pgfdeclarelayer{nodelayer}
\pgfsetlayers{edgelayer,nodelayer,main}

\tikzstyle{none}=[inner sep=0pt]
\tikzstyle{rn}=[circle,fill=Red,draw=Black,line width=0.8 pt]
\tikzstyle{gn}=[circle,fill=Lime,draw=Black,line width=0.8 pt]
\tikzstyle{yn}=[circle,fill=Yellow,draw=Black,line width=0.8 pt]
\tikzstyle{empty}=[circle,fill=White,draw=Black]
\tikzstyle{bw} = [rectangle, draw, fill=blue!20, 
    text width=4em, text centered, rounded corners, minimum height=2em]
    
    \newcommand{\CcNote}[1]{% longname
	This work is licensed under the \textit{Creative Commons #1 3.0 License}.%
}
\newcommand{\CcImageBy}[1]{%
	\includegraphics[scale=#1]{creative_commons/cc_by_30.pdf}%
}
\newcommand{\CcImageSa}[1]{%
	\includegraphics[scale=#1]{creative_commons/cc_sa_30.pdf}%
}
\newcommand{\CcImageNc}[1]{%
	\includegraphics[scale=#1]{creative_commons/cc_nc_30.pdf}%
}
\newcommand{\CcGroupBySa}[2]{% zoom, gap
	\CcImageBy{#1}\hspace*{#2}\CcImageNc{#1}\hspace*{#2}\CcImageSa{#1}%
}
\newcommand{\CcLongnameByNcSa}{Attribution-NonCommercial-ShareAlike}

\newenvironment{changemargin}[1]{% 
  \begin{list}{}{% 
    \setlength{\topsep}{0pt}% 
    \setlength{\leftmargin}{#1}% 
    \setlength{\rightmargin}{1em}
    \setlength{\listparindent}{\parindent}% 
    \setlength{\itemindent}{\parindent}% 
    \setlength{\parsep}{\parskip}% 
  }% 
  \item[]}{\end{list}} 



\usepackage{alltt}

\title{Lecture 19 --- Debugging}

\author{Jeff Zarnett \& Patrick Lam \\ \small \texttt{jzarnett@uwaterloo.ca} \& \texttt{p.lam@ece.uwaterloo.ca}}
\institute{Department of Electrical and Computer Engineering \\[-1ex]
  University of Waterloo}
\date{\today}

\begin{document}

\begin{frame}
  \titlepage


\end{frame}

\begin{frame}
\frametitle{Debugging}
\begin{changemargin}{1cm}
After testing, it's time to look at debugging. 

Programmers once believed it couldn't be done systematically.

The ``Richard Feynman Method'':
\begin{enumerate}
	\item Write down the problem.
	\item Think very hard.
	\item Write down the answer.
\end{enumerate}

Not very realistic, especially for code someone else wrote.

\mnote{This approach might happen sometimes, but it requires deep insight to the code and a sudden burst of brilliance to get it done. That's not very realistic, especially if you are debugging code you didn't write in the first place \cite{dgtd}.}

\end{changemargin}
\end{frame}


\begin{frame}
\frametitle{Two Kinds of Software}
\begin{changemargin}{1cm}
There are two kinds of software:

Those that are so simple they obviously have no bugs.

Those that are so complex they have no obvious bugs.

\mnote{ The first category of programs is smaller than we might think. Look at our labs - the amount of code we write is small, and yet, I can say with certainty that every implementation, including mine, has bugs.}

More simply stated: all software has bugs.

\end{changemargin}
\end{frame}

\begin{frame}
\frametitle{The Systematic Approach}
\begin{changemargin}{1cm}
There are two kinds of software:

Those that are so simple they obviously have no bugs.

Those that are so complex they have no obvious bugs.

\mnote{ The first category of programs is smaller than we might think. Look at our labs - the amount of code we write is small, and yet, I can say with certainty that every implementation, including mine, has bugs.}

More simply stated: all software has bugs.

\end{changemargin}
\end{frame}

\begin{frame}
\frametitle{Debugging Strategy}

\begin{changemargin}{1cm}
\Large
Based on the scientific method.
\end{changemargin}

\begin{changemargin}{2.3cm} \setlength{\itemsep}{1.5em}
\large
\begin{enumerate}
\item Observe a failure.
\item Invent a hypothesis.
\item Make predictions.
\item Test the predictions using experiments and observations.
\begin{itemize}
\item Correct? Refine the hypothesis.
\item Wrong? Try again with a new hypothesis.
\end{itemize}
\item Repeat steps 3 and 4 as needed.
\end{enumerate}

~\\
Note: be explicit!
\end{changemargin}

\end{frame}

\begin{frame}
\frametitle{Observing a failure}

\begin{changemargin}{1cm}
\Large

Basic problem: output is not as expected.\\[1em]

Write down:
\begin{itemize}
\item the circumstances;
\item expected output; and,
\item actual output.
\end{itemize}

~\\
\structure{Example}: When I enter a negative number like -5 (input) into my app (circumstances), it loops infinitely (output).

\end{changemargin}

\end{frame}

\begin{frame}
\frametitle{Why is this happening?}

\begin{changemargin}{1cm}
\Large Make a hypothesis.\\[1em]

Your hypothesis guesses at a cause of the failure, consistent with 
the observations.\\[1em]

\large
\structure{Example}: The stopping condition in my program is when
{\tt counter == 0}, which never occurs as I'm decrementing 
{\tt counter} and it is negative.
\end{changemargin}

\end{frame}

\begin{frame}
\frametitle{Make predictions}

\begin{changemargin}{1cm}
\Large
What else would happen if your prediction was correct?\\[1em]

\large
\structure{Example}: My program would also loop infinitely on
an input of 1.5, as decrementing 1.5 would also never hit 0.

\end{changemargin}

\end{frame}

\begin{frame}
\frametitle{Test and Refine/Discard}

\begin{changemargin}{1cm}
\Large
Perform an experiment to see if your hypothesis is correct.

\large
\begin{itemize}
\item Yes? OK, you can refine the hypothesis.
\item No? Try something else.
\end{itemize}
~\\

\structure{Example}: Yes, feeding 1.5 to my app loops 30-odd times
before it crashes. Hypothesis seems correct.

\end{changemargin}

\end{frame}

\begin{frame}
\frametitle{Test and Refine/Discard}
\begin{changemargin}{1cm}
\Large
When testing your hypothesis, also try something that would \emph{disprove} your theory.

It's a natural human tendency (called confirmation bias).
\end{changemargin}

\end{frame}


\begin{frame}
\frametitle{Repeat as Needed}

\begin{changemargin}{1cm}

\Large
Until you have an actionable hypothesis, continue to refine or
discard it, conducting experiments along the way.

\end{changemargin}

\end{frame}

\begin{frame}
\frametitle{Fix the problem}

\begin{changemargin}{1cm}
\Large
Modify the code so that the failure can no longer occur.\\[1em]

\large
\structure{Example}: Instead of checking {\tt counter == 0},
check {\tt counter > 0}.
\end{changemargin}

\end{frame}


\begin{frame}
\frametitle{Bug Localization}

\begin{changemargin}{1cm}

\Large
Key part of the hypothesis: \\
~~~~\structure{where} is the problem?\\[1em]

\large
\structure{Example}: The failure was caused at an incorrect 
test for {\tt counter == 0}.\\[1em]

\Large
Isolate the failure to
a specific subsystem or module.\\[1em]

Can swap out with known-good versions of modules as an experiment.

\end{changemargin}

\end{frame}

\begin{frame}
\frametitle{Debugging Tactics}

\begin{changemargin}{1cm}
\Large
Four key tactics:
\begin{itemize}
\item Code review.
\item Code instrumentation.
\item Single-step execution.
\item Take a Break.
\end{itemize}

\end{changemargin}

\end{frame}

\begin{frame}
\frametitle{Take a Break}

\begin{changemargin}{1cm}


Take a break?!

Sometimes when you're stumped, a break is what you need.

Even taking 15 minutes to get coffee can help.\\
\quad Other times, going home and getting a good night's sleep.


Your subconscious continues to work on the problem.\\
\quad The answer might come to you when you don't expect it.


\end{changemargin}

\end{frame}

\begin{frame}
\frametitle{Code Review}

\begin{changemargin}{1cm}
\Large

Just stare at the code.\\[1em]

\begin{itemize}
\item I find this most effective.
\item You need to have a good idea of where the fault lies.
\item Get a friend to help; rubber duck debugging?
\end{itemize}

\end{changemargin}

\end{frame}

\begin{frame}
\frametitle{Code instrumentation}

\begin{changemargin}{1cm}
\Large

Works very well with the strategy above.

\begin{itemize}
\item Use {\tt print} or {\tt Log.d} statements to get
information on program state.
\item Verify hypotheses based on this information.
\end{itemize}

\end{changemargin}

\end{frame}

\begin{frame}
\frametitle{Single-step execution}

\begin{changemargin}{1cm}
\Large

Use a debugger to manually inspect program state.

\begin{itemize}
\item Low-level view of variable contents.
\item Easy to get bogged down.
\end{itemize}

\end{changemargin}

\end{frame}

\begin{frame}
\frametitle{Tactics for Bug Localization}

\begin{changemargin}{1cm}
\Large

\begin{itemize}\setlength{\itemsep}{1em}
\item Supply different inputs;
\item Instrument the program;
\item Run the program;
\item Set breakpoints;
\item Examine internal state.
\end{itemize}

\end{changemargin}

\end{frame}


\begin{frame}
\frametitle{Assertions}

\begin{changemargin}{1cm}
\Large

\structure{A statement about the world.}

\normalsize
\begin{itemize}
\item Should always be true.

\item Not really for debugging; more for in-line documentation. 

\item Aids debugging when it fails---something to fix.
\end{itemize}

\end{changemargin}

\end{frame}

\begin{frame}[fragile]
\frametitle{Assertion Examples}

\begin{changemargin}{.5cm}

\begin{verbatim}
  /* example 1: i is odd */
  if (i % 2 == 0) 
  { 
    ... 
  } 
  else 
  { 
    assert i % 2 == 1; 
    ...
  }

  /* example 2: doubly-linked list */
  assert this.next.prev == this : 
    (this+" fails doubly-linked node invariant");
\end{verbatim}

\end{changemargin}

\end{frame}


\begin{frame}
\frametitle{Assertion Gotcha}

\Large
\begin{center}
\structure{An assertion should never have a side effect.}
\end{center}

\end{frame}

\begin{frame}
\frametitle{Single-step execution}

\begin{changemargin}{1cm}
\Large
Helps with both error localization and hypothesis testing.\\[1em]

Interactively monitor and change program values as the
program is executing.
\end{changemargin}

\end{frame}

\begin{frame}
\frametitle{Problem with Single-step execution}

\begin{changemargin}{1cm}
\Large
Too many steps!\\[1em]

\uncover<2>{Solution: \alert{breakpoints}.}

\end{changemargin}

\end{frame}

\begin{frame}
\frametitle{Kinds of breakpoints}

\begin{changemargin}{1cm}

\Large
\begin{itemize}
\item Line breakpoints
\item Exception breakpoints
\item Watchpoints
\item Method breakpoints
\end{itemize}

\end{changemargin}

\end{frame}


\begin{frame}
\frametitle{Eclipse Demo}

\begin{changemargin}{1cm}

To close out I'll show you a bit of debugging in Eclipse.
\end{changemargin}

\end{frame}


\end{document}
