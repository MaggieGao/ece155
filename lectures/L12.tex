\documentclass[letterpaper,10pt]{article}

\usepackage{titling}
\usepackage{listings}
\usepackage{url}
\usepackage{setspace}
\usepackage{subfig}
\usepackage{sectsty}
\usepackage{pdfpages}
\usepackage{colortbl}
\usepackage{multirow}
\usepackage{relsize}
\usepackage{amsmath}
\usepackage{fancyvrb}
\usepackage{amsmath,amssymb,amsthm,graphicx,xspace}
\usepackage[titlenotnumbered,noend,noline]{algorithm2e}
\usepackage[compact]{titlesec}
\usepackage[default]{droidserif}
\usepackage[T1]{fontenc}
\usepackage{tikz}
\usetikzlibrary{arrows,automata,shapes,trees,matrix,chains,scopes,positioning,calc}
\tikzstyle{block} = [rectangle, draw, fill=blue!20, 
    text width=2.5em, text centered, rounded corners, minimum height=2em]
\tikzstyle{bw} = [rectangle, draw, fill=blue!20, 
    text width=4em, text centered, rounded corners, minimum height=2em]

\definecolor{namerow}{cmyk}{.40,.40,.40,.40}
\definecolor{namecol}{cmyk}{.40,.40,.40,.40}

\let\LaTeXtitle\title
\renewcommand{\title}[1]{\LaTeXtitle{\textsf{#1}}}


\newcommand{\handout}[5]{
  \noindent
  \begin{center}
  \framebox{
    \vbox{
      \hbox to 5.78in { {\bf ECE155: Engineering Design with Embedded Systems } \hfill #2 }
      \vspace{4mm}
      \hbox to 5.78in { {\Large \hfill #4  \hfill} }
      \vspace{2mm}
      \hbox to 5.78in { {\em #3 \hfill} }
    }
  }
  \end{center}
  \vspace*{4mm}
}

\newcommand{\lecture}[3]{\handout{#1}{#2}{#3}{Lecture #1}}
\newcommand{\tuple}[1]{\ensuremath{\left\langle #1 \right\rangle}\xspace}

\addtolength{\oddsidemargin}{-1.000in}
\addtolength{\evensidemargin}{-0.500in}
\addtolength{\textwidth}{2.0in}
\addtolength{\topmargin}{-1.000in}
\addtolength{\textheight}{1.75in}
\addtolength{\parskip}{\baselineskip}
\setlength{\parindent}{0in}
\renewcommand{\baselinestretch}{1.5}
\newcommand{\term}{Spring 2014}

\singlespace



\begin{document}

\lecture{ 12 ---Computational Decision-Making}{\term}{Patrick Lam \& Jeff Zarnett}

\section*{Decision-Making}
Let's start with a general process for 
making decisions. 

\begin{enumerate}
\item List all choices or courses of action.
\item List all criteria (attributes) that will be used to evaluate the choices.
\item Compare the two lists and remove impractical choices.
\item Evaluate the advantages \& disadvantages of each remaining choice
according to all of the criteria.
\end{enumerate}

This doesn't really say how to evaluate the advantages and disadvantages
systematically, though. 

\subsection*{Computational Decision-Making}
Our goal is to quantify the decision-making process. We'll do so by assigning 
a numerical \emph{weight} to each criterion, according to its importance, and a
numerical \emph{score} for each choice according to each criterion.
Together, these allow us to compute the value of a payoff function for
each choice.

Assume that there are $m$ choices and $n$ criteria. 
\begin{enumerate}
\item Decide on the importance of each criterion, and assign a weight
  $w_j$ to each criterion. Make sure that $\sum_{j=1}^n w_j = 1$ (or
  100\%). Larger weights are more important.
\item For each choice $i$ and criterion $j$, assign a score $p_{ij}$
  which reflects the extent to which $i$ satisfies $j$. This should be
  a number between 0 and 1.
\end{enumerate}
This data will allow us to compute scores $s_{ij}$ and a payoff
function $f_i$ for each alternative and to choose the alternative with
the largest expected payoff:
\[ s_{ij} = p_{ij} w_j; \qquad f_i = \sum_{j=1}^n s_{ij} = \sum_{j=1}^n p_{ij} w_j \]
(which is just the dot product of the weights and the scores).

Choose the alternative with the largest payoff $f_i$.

\paragraph{Normalizing scores.} As an alternative to choosing scores
$p_{ij} \in [0, 1]$, you can instead normalize the scores,
as follows.
\begin{enumerate}
\item Assign a real number $c_{ij}$ for each alternative $i$ and criterion $j$.
Be sure to use the same units for all of the different alternatives of a 
single criterion. You can use different units for different criteria.
\item For each criterion, calculate
\[ C_j = \max \{ |c_{1j}|, |c_{2j}|, \ldots, |c_{mj}| \}, \]
so that
\[ p_{ij} = \frac{c_{ij}}{C_j}. \]
\item The payoff function is then given by:
\[ f_i = \sum_{j=1}^n p_{ij} w_j = \sum_{j=1}^n \frac{c_{ij}}{C_j}  w_j. \]
\end{enumerate}

\subsection*{Example}
Let's consider the question: ``How should I get to Montreal?''

The possible answers I'll consider are:
\begin{itemize}
\item Train
\item Personal automobile
\item Airplane
\end{itemize}

In Step 1, we will consider three criteria:
\begin{itemize}
\item Cost
\item Travel time
\item Schedule Flexibility (i.e. when can I leave?)
\end{itemize}
Let's assign weights to these criteria. Step 2:
\begin{itemize}
\item Pretend you have lots of money. Let's estimate $w_1 = 0.2$.
\item Time is important, so $w_2 = 0.4$.
\item Schedule Flexibility is important; let $w_3 = 0.4$.
\end{itemize}
(Yes, these are arbitrary. We'll see just how much these arbitrary
decisions affect the outcome.)

Now, we need to decide how we'll assign scores. Step 3:
\begin{itemize}
\item Cost: dollars (lower = better);
\item Time: hours (lower = better);
\item Schedule Flexibility: qualitative assessment (higher = better)
\end{itemize}
All of the scores need to point the same way, so we'll invert our
schedule flexibility score before computing.

Let's make up some raw data.

\begin{center}
\begin{tabular}{lllll}
Criterion & Train & Car & Plane & Category Winner \\
Cost & \$102 & \$206 & \$191 & Train \\
Time & 7.7 hrs & 6.5 hrs & 3.5 hrs & Plane \\
Schedule Flexibility & 0.5 & 1.0 & 0.9 & Car
\end{tabular}
\end{center}

The cost for driving is estimated at \$206 by the calculation of 642km at \$0.321/km, per \url{http://www.caa.ca/documents/DrivingCostsBrochure-jan09-eng-v3.pdf}. Plane and train costs are based on the ticket purchase prices.

{\sf What factors do these data not capture?}\\[2em]

Now we can build a decision matrix:

{\Large
\begin{center}
\begin{tabular}{l|c|rrr|rrr|rrr}
Criterion & Weight & \multicolumn{9}{c}{Alternatives} \\
$(n=3)$ & $w_i$ (\%) & \multicolumn{3}{c|}{Train} & \multicolumn{3}{c|}{Car} & \multicolumn{3}{c}{Plane} \\
&& $c_{1j}$ & $p_{1j}$& $s_{1j}$ & $c_{2j}$ & $p_{2j}$ & $s_{2j}$ & $c_{3j}$ & $p_{3j}$ & $s_{3j}$ \\  \hline
Cost & 20 & 102 & 0.5 & & 206 & 1.0 & & 191 & 0.9 &  \\
Time & 40 & 7.7 & 1.0 & & 6.5 & 0.85 & & 3.5 & 0.45 \\
Flexibility & 40 & 0.5&0.5 &&0.0 &0.0 &&0.1 &0.1& \\ \hline
\multicolumn{2}{c|}{Totals $f_j$} & 
\end{tabular}
\end{center}
}
Note that, in this matrix, lower is better. It may make sense to make
higher better, depending on how your criteria are set up.

\paragraph{Potential Flaws.} Did we get an optimal decision? Let's
look at threats to validity of the analysis to see how robust our decision
is.

\begin{itemize}
\item (Not an issue here.) In some cases, your scores might be stuck in a
small subrange of the possible range, which affects scores and hence values.
%\item The values for each criteria have different ranges: flexibility
%is in $[0, 0.5]$ while cost is in $[0.5, 1.0]$ and time is in $[0.45, 1.0]$.
\item The values for flexibility are subjective, and the way you assign
numbers to the options changes the outcome.
\item The cost depends on what you include; for instance, gas alone would
be \$72.50, and it would be less if you carpooled.
\item The weights are subjective.
\end{itemize}

\paragraph{Sensitivity Analysis.} To investigate the impact of the
potential flaws, we can use a \emph{sensitivity analysis}, which is
a study of how variations influence the outcome of a mathematical model.

You can do a poor person's sensitivity analysis by performing what-if
calculations: simply replace the original value with some other value
and repeat the analysis. There can be issues if the scores are not
independent.

There are also more complex techniques for sensitivity analysis, which
we won't talk about. Let's instead try revising the scores. 
For instance, the train used to always be late, so you might assign 9
instead of 7.7. (It's mostly on time recently.) On the other hand, you
might want to get to the airport slightly earlier than me, so you might also
change the time for plane to 5. This gives:

{\Large
\begin{center}
\begin{tabular}{l|c|rrr|rrr|rrr}
Criterion & Weight & \multicolumn{9}{c}{Alternatives} \\
$(n=3)$ & $w_i$ (\%) & \multicolumn{3}{c|}{Train} & \multicolumn{3}{c|}{Car} & \multicolumn{3}{c}{Plane} \\
&& $c_{i1}$ & $p_{i1}$& $s_{i1}$ & $c_{i2}$ & $p_{i2}$ & $s_{i2}$ & $c_{i3}$ & $p_{i3}$ & $s_{i3}$ \\  \hline
Cost & 20 & 102 & 0.5 & & 206 & 1.0 & & 191 & 0.9 &  \\
Time & 40 & 9 & 1.0 & & 6.5 & 0.72 & & 5 & 0.56 \\
Flexibility & 40 & 0.5&0.5 &&0.0 &0.0 &&0.1 &0.1& \\ \hline
\multicolumn{2}{c|}{Totals $f_j$} & 
\end{tabular}
\end{center}
}

{\sf Does this change the best decision?}
\vspace{2em}

Or, you might be more cost-sensitive and less time-sensitive:
{\Large
\begin{center}
\begin{tabular}{l|c|rrr|rrr|rrr}
Criterion & Weight & \multicolumn{9}{c}{Alternatives} \\
$(n=3)$ & $w_i$ (\%) & \multicolumn{3}{c|}{Train} & \multicolumn{3}{c|}{Car} & \multicolumn{3}{c}{Plane} \\
&& $c_{i1}$ & $p_{i1}$& $s_{i1}$ & $c_{i2}$ & $p_{i2}$ & $s_{i2}$ & $c_{i3}$ & $p_{i3}$ & $s_{i3}$ \\  \hline
Cost & 40 & 102 & 0.5 & & 206 & 1.0 & & 191 & 0.9 &  \\
Time & 20 & 7.7 & 1.0 & & 6.5 & 0.85 & & 3.5 & 0.45 \\
Flexibility & 40 & 0.5&0.5 &&0.0 &0.0 &&0.1 &0.1& \\ \hline
\multicolumn{2}{c|}{Totals $f_j$} & 
\end{tabular}
\end{center}
}

{\sf What happens now?}
\vspace{2em}

\paragraph{Other references.}
Previous notes contained an example (``Where should I go to
University?'') and a further reference to \emph{Introduction to
  Professional Engineering in Canada}, Example 15.3
(pp. 232--233). You can also see some aspects of computational
decision-making if you look at the consideration of Waterloo Region's
Rapid Transit alternatives, although there's no actual numbers.  

\subsection*{Discussion}
Computational decision-making systematically includes diverse factors
to arrive at a ``best'' decision.

\paragraph{Disadvantage.} You have to estimate, numerically, the 
importance (weight) of each criterion and the goodness (score) of each 
alternative with respect to each criterion. It may be infeasible to
come up with good estimates, and estimates may just reflect your
pre-conceived notions. It also doesn't give you new alternatives.

\paragraph{Advantage.} Using this method, you can ensure that you give
each alternative sufficient consideration. You don't necessarily have to
pick the alternative with the highest number, because your scores or
coefficients might have been imprecise.

\paragraph{Work Term Report Pro Tip.} Computational decision-making charts are valuable in your work reports. They provide a consistent way of evaluating alternatives and justifying why you have chosen a particular option as the best.

\subsection*{Further Example}
This example came from the Spring 2014 midterm exam. You are a conscientious student who works hard, but you are also a fan of heavy metal bands whose names begin with the letter \textit{M}. You think that you can manage to go to one concert this summer without having a negative impact on your studies. There are three concerts scheduled in your general area, and you have researched the subject and composed the following raw data.

\begin{table}[h]
\begin{center}
	\begin{tabular}{c|c|c|c}
	\textbf{Criterion} & \textbf{Manowar} & \textbf{Megadeth} & \textbf{Metallica}\\
		\hline
		Ticket Cost (\$) & 100 & 125 & 75\\
		Travel Time (h) & 1.5 & 3 & 3 \\
		Show Reviews & 8/10 & 10/10 & 6/10 \\
	\end{tabular}
\end{center}
\end{table}

You also estimate the following weights:

\begin{table}[h]
\begin{center}
	\begin{tabular}{c|c}
	\textbf{Criterion} & Weight (\%)\\
		\hline
		Ticket Cost & 50\\
		Travel Time & 20\\
		Show Reviews & 30
	\end{tabular}
\end{center}
\end{table}

Complete a computational decision-making chart, and indicate which of the three concerts you plan to attend.


\paragraph{Solution.}
Show reviews is the only criterion where higher is better, so it must be inverted. Also acceptable is inverting both ticket code and travel time (although not shown here).

{\Large
\begin{center}
\begin{tabular}{l|c|r|r|r|r|r|r|r|r|r|}
Criterion & Weight & \multicolumn{9}{c}{Alternatives} \\
$(n=3)$ & $w_i$ (\%) & \multicolumn{3}{c|}{Manowar} & \multicolumn{3}{c|}{Megadeth} & \multicolumn{3}{c}{Metallica} \\
&& $c_{i1}$ & $p_{i1}$& $s_{i1}$ & $c_{i2}$ & $p_{i2}$ & $s_{i2}$ & $c_{i3}$ & $p_{i3}$ & $s_{i3}$ \\  \hline
Ticket Cost & 50 & 100 & 0.8 & 40 & 125 & 1 & 50 & 75 & 0.6 & 30 \\
Travel Time & 20 & 1.5 & 0.5 & 10 & 3   & 1 & 20 & 3  & 1   & 20 \\
Reviews     & 30 & 2 & 0.2 & 6    & 0   & 0 & 0  & 4  & 0.4 & 12\\ 
\hline
\multicolumn{2}{c|}{Totals $f_j$} & \multicolumn{3}{r}{56} & \multicolumn{3}{r}{70} & \multicolumn{3}{r}{62}
\end{tabular}
\end{center}
}



\bibliographystyle{alpha}
\bibliography{155}


\end{document}