
\documentclass[letterpaper,hide notes,xcolor={table,svgnames},pdftex]{beamer}
\def\showexamples{t}


%\usepackage[svgnames]{xcolor}

%% Demo talk
%\documentclass[letterpaper,notes=show]{beamer}

\usecolortheme{crane}
\setbeamertemplate{navigation symbols}{}

\usetheme{MyPittsburgh}
%\usetheme{Frankfurt}

%\usepackage{tipa}

\usepackage{hyperref}
\usepackage{graphicx,xspace}
\usepackage[normalem]{ulem}

\newcommand\SF[1]{$\bigstar$\footnote{SF: #1}}



\newcounter{tmpnumSlide}
\newcounter{tmpnumNote}

% old question code
%\newcommand\question[1]{{$\bigstar$ \small \onlySlide{2}{#1}}}
% \newcommand\nquestion[1]{\ifdefined \presentationonly \textcircled{?} \fi \note{\par{\Large \textbf{?}} #1}}
% \newcommand\nanswer[1]{\note{\par{\Large \textbf{A}} #1}}


 \newcommand\mnote[1]{%
   \addtocounter{tmpnumSlide}{1}
   \ifdefined\showcues {~\tiny\fbox{\arabic{tmpnumSlide}}}\fi
   \note{\setlength{\parskip}{1ex}\addtocounter{tmpnumNote}{1}\textbf{\Large \arabic{tmpnumNote}:} {#1\par}}}

\newcommand\mmnote[1]{\note{\setlength{\parskip}{1ex}#1\par}}

%\newcommand\mnote[2][]{\ifdefined\handoutwithnotes {~\tiny\fbox{#1}}\fi
% \note{\setlength{\parskip}{1ex}\textbf{\Large #1:} #2\par}}

%\newcommand\mnote[2][]{{\tiny\fbox{#1}} \note{\setlength{\parskip}{1ex}\textbf{\Large #1:} #2\par}}

\newcommand\mquestion[2]{{~\color{red}\fbox{?}}\note{\setlength{\parskip}{1ex}\par{\Large \textbf{?}} #1} \note{\setlength{\parskip}{1ex}\par{\Large \textbf{A}} #2\par}\ifdefined \presentationonly \pause \fi}

\newcommand\blackboard[1]{%
\ifdefined   \showblackboard
  {#1}
  \else {\begin{center} \fbox{\colorbox{blue!30}{%
         \begin{minipage}{.95\linewidth}%
           \hspace{\stretch{1}} Some space intentionally left blank; done at the blackboard.%
         \end{minipage}}}\end{center}}%
         \fi%
}



%\newcommand\q{\tikz \node[thick,color=black,shape=circle]{?};}
%\newcommand\q{\ifdefined \presentationonly \textcircled{?} \fi}

\usepackage{listings}
\lstset{%
  keywordstyle=\bfseries,
  aboveskip=15pt,
  belowskip=15pt,
  captionpos=b,
  identifierstyle=\ttfamily,
  escapeinside={(*@}{@*)},
  stringstyle=\ttfamiliy,
  frame=lines,
  numbers=left, basicstyle=\scriptsize, numberstyle=\tiny, stepnumber=0, numbersep=2pt}

\usepackage{siunitx}
\newcommand\sius[1]{\num[group-separator = {,}]{#1}\si{\micro\second}}
\newcommand\sims[1]{\num[group-separator = {,}]{#1}\si{\milli\second}}
\newcommand\sins[1]{\num[group-separator = {,}]{#1}\si{\nano\second}}
\sisetup{group-separator = {,}, group-digits = true}

%% -------------------- tikz --------------------
\usepackage{tikz}
\usetikzlibrary{positioning}
\usetikzlibrary{arrows,backgrounds,automata,decorations.shapes,decorations.pathmorphing,decorations.markings,decorations.text}

\tikzstyle{place}=[circle,draw=blue!50,fill=blue!20,thick, inner sep=0pt,minimum size=6mm]
\tikzstyle{transition}=[rectangle,draw=black!50,fill=black!20,thick, inner sep=0pt,minimum size=4mm]

\tikzstyle{block}=[rectangle,draw=black, thick, inner sep=5pt]
\tikzstyle{bullet}=[circle,draw=black, fill=black, thin, inner sep=2pt]

\tikzstyle{pre}=[<-,shorten <=1pt,>=stealth',semithick]
\tikzstyle{post}=[->,shorten >=1pt,>=stealth',semithick]
\tikzstyle{bi}=[<->,shorten >=1pt,shorten <=1pt, >=stealth',semithick]

\tikzstyle{mut}=[-,>=stealth',semithick]

\tikzstyle{treereset}=[dashed,->, shorten >=1pt,>=stealth',thin]

\usepackage{ifmtarg}
\usepackage{xifthen}
\makeatletter
% new counter to now which frame it is within the sequence
\newcounter{multiframecounter}
% initialize buffer for previously used frame title
\gdef\lastframetitle{\textit{undefined}}
% new environment for a multi-frame
\newenvironment{multiframe}[1][]{%
\ifthenelse{\isempty{#1}}{%
% if no frame title was set via optional parameter,
% only increase sequence counter by 1
\addtocounter{multiframecounter}{1}%
}{%
% new frame title has been provided, thus
% reset sequence counter to 1 and buffer frame title for later use
\setcounter{multiframecounter}{1}%
\gdef\lastframetitle{#1}%
}%
% start conventional frame environment and
% automatically set frame title followed by sequence counter
\begin{frame}%
\frametitle{\lastframetitle~{\normalfont(\arabic{multiframecounter})}}%
}{%
\end{frame}%
}
\makeatother

\makeatletter
\newdimen\tu@tmpa%
\newdimen\ydiffl%
\newdimen\xdiffl%
\newcommand\ydiff[2]{%
    \coordinate (tmpnamea) at (#1);%
    \coordinate (tmpnameb) at (#2);%
    \pgfextracty{\tu@tmpa}{\pgfpointanchor{tmpnamea}{center}}%
    \pgfextracty{\ydiffl}{\pgfpointanchor{tmpnameb}{center}}%
    \advance\ydiffl by -\tu@tmpa%
}
\newcommand\xdiff[2]{%
    \coordinate (tmpnamea) at (#1);%
    \coordinate (tmpnameb) at (#2);%
    \pgfextractx{\tu@tmpa}{\pgfpointanchor{tmpnamea}{center}}%
    \pgfextractx{\xdiffl}{\pgfpointanchor{tmpnameb}{center}}%
    \advance\xdiffl by -\tu@tmpa%
}
\makeatother
\newcommand{\copyrightbox}[3][r]{%
\begin{tikzpicture}%
\node[inner sep=0pt,minimum size=2em](ciimage){#2};
\usefont{OT1}{phv}{n}{n}\fontsize{4}{4}\selectfont
\ydiff{ciimage.south}{ciimage.north}
\xdiff{ciimage.west}{ciimage.east}
\ifthenelse{\equal{#1}{r}}{%
\node[inner sep=0pt,right=1ex of ciimage.south east,anchor=north west,rotate=90]%
{\raggedleft\color{black!50}\parbox{\the\ydiffl}{\raggedright{}#3}};%
}{%
\ifthenelse{\equal{#1}{l}}{%
\node[inner sep=0pt,right=1ex of ciimage.south west,anchor=south west,rotate=90]%
{\raggedleft\color{black!50}\parbox{\the\ydiffl}{\raggedright{}#3}};%
}{%
\node[inner sep=0pt,below=1ex of ciimage.south west,anchor=north west]%
{\raggedleft\color{black!50}\parbox{\the\xdiffl}{\raggedright{}#3}};%
}
}
\end{tikzpicture}
}


%% --------------------

%\usepackage[excludeor]{everyhook}
%\PushPreHook{par}{\setbox0=\lastbox\llap{MUH}}\box0}

%\vspace*{\stretch{1}

%\setbox0=\lastbox \llap{\textbullet\enskip}\box0}

\setlength{\parskip}{\fill}

\newcommand\noskips{\setlength{\parskip}{1ex}}
\newcommand\doskips{\setlength{\parskip}{\fill}}

\newcommand\xx{\par\vspace*{\stretch{1}}\par}
\newcommand\xxs{\par\vspace*{2ex}\par}
\newcommand\tuple[1]{\langle #1 \rangle}
\newcommand\code[1]{{\sf \footnotesize #1}}
\newcommand\ex[1]{\uline{Example:} \ifdefined \presentationonly \pause \fi
  \ifdefined\showexamples#1\xspace\else{\uline{\hspace*{2cm}}}\fi}

\newcommand\ceil[1]{\lceil #1 \rceil}


\AtBeginSection[]
{
   \begin{frame}
       \frametitle{Outline}
       \tableofcontents[currentsection]
   \end{frame}
}



\pgfdeclarelayer{edgelayer}
\pgfdeclarelayer{nodelayer}
\pgfsetlayers{edgelayer,nodelayer,main}

\tikzstyle{none}=[inner sep=0pt]
\tikzstyle{rn}=[circle,fill=Red,draw=Black,line width=0.8 pt]
\tikzstyle{gn}=[circle,fill=Lime,draw=Black,line width=0.8 pt]
\tikzstyle{yn}=[circle,fill=Yellow,draw=Black,line width=0.8 pt]
\tikzstyle{empty}=[circle,fill=White,draw=Black]
\tikzstyle{bw} = [rectangle, draw, fill=blue!20, 
    text width=4em, text centered, rounded corners, minimum height=2em]
    
    \newcommand{\CcNote}[1]{% longname
	This work is licensed under the \textit{Creative Commons #1 3.0 License}.%
}
\newcommand{\CcImageBy}[1]{%
	\includegraphics[scale=#1]{creative_commons/cc_by_30.pdf}%
}
\newcommand{\CcImageSa}[1]{%
	\includegraphics[scale=#1]{creative_commons/cc_sa_30.pdf}%
}
\newcommand{\CcImageNc}[1]{%
	\includegraphics[scale=#1]{creative_commons/cc_nc_30.pdf}%
}
\newcommand{\CcGroupBySa}[2]{% zoom, gap
	\CcImageBy{#1}\hspace*{#2}\CcImageNc{#1}\hspace*{#2}\CcImageSa{#1}%
}
\newcommand{\CcLongnameByNcSa}{Attribution-NonCommercial-ShareAlike}

\newenvironment{changemargin}[1]{% 
  \begin{list}{}{% 
    \setlength{\topsep}{0pt}% 
    \setlength{\leftmargin}{#1}% 
    \setlength{\rightmargin}{1em}
    \setlength{\listparindent}{\parindent}% 
    \setlength{\itemindent}{\parindent}% 
    \setlength{\parsep}{\parskip}% 
  }% 
  \item[]}{\end{list}} 




\title{Lecture 13 --- Android III}

\author{Patrick Lam \\ \small \texttt{p.lam@ece.uwaterloo.ca}}
\institute{Department of Electrical and Computer Engineering \\
  University of Waterloo}
\date{\today}


\begin{document}

\begin{frame}
  \titlepage
\end{frame}

\part{Android Persistent Storage}

\begin{frame}
\partpage
\end{frame}




\begin{frame}
\frametitle{Storing Data}

\begin{changemargin}{1cm}
So far, we haven't seen any ways to store data so that you can re-load
it across different instances of your application.


Four+ options:
\begin{itemize}
\item Shared Preferences
\item Files: Internal and External Storage
\item SQLite
\item on the Internet
\end{itemize}
\end{changemargin}
\end{frame}

\begin{frame}
\frametitle{Shared Preferences}

\begin{changemargin}{1cm}
Key-value pairs.

Persists across user sessions, even if app gets killed.

Can be private to your application or shared across applications.
\end{changemargin}

\end{frame}


\begin{frame}[fragile]
\frametitle{Using Shared Preferences}
\begin{changemargin}{1cm}

How do we get our
hands on the shared preferences?
\begin{verbatim}
  SharedPreferences settings = getPreferences(0);
\end{verbatim}
Or, call {\tt getSharedPreferences()} and pass it a preferences file
name as the first parameter.
\end{changemargin}
\end{frame}

\begin{frame}[fragile]
\frametitle{Reading Data from Shared Preferences}
\begin{changemargin}{1cm}
Now that you have a {\tt SharedPreferences} object, you can get data from it:

\begin{verbatim}
  v = settings.getString("textFieldValue", "");
\end{verbatim}
Retrieves the value of {\tt String} preference {\tt textFieldValue}. 
  
  If no value is present, the {\tt getString} call
returns the default value {\tt ""}, an empty string.

\end{changemargin}
\end{frame}


\begin{frame}[fragile]
\frametitle{Writing to Shared Preferences}
\begin{changemargin}{1cm}

It's a bit more complicated to write to the {\tt SharedPreferences}.

You have to first get a {\tt SharedPreferences.Editor}.

Then ``commit'' it once you're done with the changes.

\begin{verbatim}
  SharedPreferences settings = getPreferences(0);
  SharedPreferences.Editor editor = settings.edit();
  editor.putString("textFieldValue", newFieldValue);
  editor.commit();
\end{verbatim}


\end{changemargin}
\end{frame}


\begin{frame}
\frametitle{Files: Internal \& External Storage}

\begin{changemargin}{1cm}
Android phones have internal storage (small) and external storage
(larger).

The commands to write to them are more or less the same.

\end{changemargin}
\end{frame}

\begin{frame}
\frametitle{Internal Storage}
\begin{changemargin}{1cm}

All Android devices have internal storage. 

It is generally
invisible to the user, other apps, and when the phone is mounted as
USB Mass Storage. 

When you remove an app, Android automatically
erases its internal storage. 

\end{changemargin}
\end{frame}

\begin{frame}
\frametitle{External Storage}
\begin{changemargin}{1cm}

External storage may be on an SD card (as on many phones).

It may be
enclosed in the device (as on the Nexus 7 tablet). 

Applications may access
their own external storage space, shared storage space, or even other
apps' storage space. 

However, external storage may go away anytime, since
the user could just pull out the SD card from the slot.

\end{changemargin}
\end{frame}

\begin{frame}
\frametitle{Files: External Storage}

\begin{changemargin}{1cm}
Our Lab 4 mapper code loads files from external storage in {\tt MapLoader.java}.

You can use DDMS to put files into external storage,
as well as using USB Mass Storage.

Files in external storage are world-accessible.

Warning: these files may go away anytime.
\end{changemargin}
\end{frame}

\begin{frame}[fragile]
\frametitle{Files: Checking External Storage Availability}

\begin{changemargin}{1cm}
Sometimes the media is not there, or not writable.
{\scriptsize
\begin{verbatim}
String state = Environment.getExternalStorageState();

if (Environment.MEDIA_MOUNTED.equals(state)) {
    // We can read and write the media
} else if (Environment.MEDIA_MOUNTED_READ_ONLY.equals(state)) {
    // We can only read the media
} else {
    // Something else is wrong. 
    // It may be one of many other states, but all we need
    //  to know is we can neither read nor write
}
\end{verbatim}
}
\end{changemargin}
\end{frame}

\begin{frame}[fragile]
\frametitle{Files: How MapLoader Works}
\begin{changemargin}{1cm}
It actually delegates to an XML parsing library. 

Recall that you call:
\begin{verbatim}
    NavigationalMap map = 
       MapLoader.loadMap(getExternalFilesDir(null), 
                         "Lab-room-peninsula.svg");
    mapView.setMap(map);
\end{verbatim}

The {\tt MapLoader} contains this line:
\begin{verbatim}
File map = new File(dir, filename);
\end{verbatim}
and then it builds a parser using the library call:
\begin{verbatim}
doc = docBuilder.parse(map);
\end{verbatim}
which does all the I/O.

\end{changemargin}
\end{frame}

\begin{frame}
\frametitle{Files: about getExternalFilesDir()}

\begin{changemargin}{1cm}
Android provides a number of shared directories:
\begin{itemize}
\item DIRECTORY\_MUSIC
\item DIRECTORY\_PICTURES
\item DIRECTORY\_RINGTONES
\end{itemize}
which all apps on the system can access.

\end{changemargin}


\end{frame}

\begin{frame}
\frametitle{Reading from Storage Example}
\begin{changemargin}{1cm}

The following code will read a line of text
into the {\tt String} variable~{\tt i}.

It currently reads from internal storage.

To read from external storage: \\
\quad 1) comment out the {\tt openFileInput} line; and\\
\quad 2) uncomment the {\tt new FileInputStream} line
\end{changemargin}
\end{frame}

\begin{frame}[fragile]
\frametitle{Reading from Files}
\begin{changemargin}{1cm}

{\scriptsize
\begin{verbatim}
try {
    // internal storage: 
    FileInputStream os = openFileInput("internal.txt");
    
    //InputStream os = new FileInputStream(
           new File(getExternalFilesDir(null), "external.txt"));
           
    BufferedReader br = new BufferedReader(
           new InputStreamReader(os));
    String i = br.readLine();
    // --> i contains the line you just read.
    os.close();
} catch (IOException e) {
   // Handle Exception...
}
\end{verbatim}
}

\end{changemargin}
\end{frame}


\begin{frame}[fragile]
\frametitle{Writing to Files}
\begin{changemargin}{1cm}

{\scriptsize
\begin{verbatim}
try {
    // internal storage:
    FileOutputStream os = openFileOutput("internal.txt", 
                        Context.MODE_PRIVATE);
                        
    //FileOutputStream os = new FileOutputStream(
                 new File(getExternalFilesDir(null), 
                        "external.txt"));

    PrintWriter osw = new PrintWriter(
                 new OutputStreamWriter(os));
    // --> write out the contents of string i.
    osw.println(i);
    osw.close();
} catch (IOException e) {
    // Handle Exception
}
\end{verbatim}
}


\end{changemargin}
\end{frame}

\begin{frame}
\frametitle{SQLite}

\begin{changemargin}{1cm}
SQL: Structured Query Language, \\
allows access to databases.

What's a database?

\begin{center}
\begin{tabular}{l|l|l|l}
id & front & frontFileName & back \\
17 & ``one'' & ``one.png'' & ``1'' \\
42 & ``two'' & ``two.png'' & ``2'' \\
99 & ``three'' & ``three.png'' & ``3'' \\
\end{tabular}
\end{center}
~

Basic interface: \structure{queries} on the database.

\end{changemargin}

\end{frame}

\part{Digression: Labs}
\frame{\partpage}


\begin{frame}
\frametitle{Lab 2: Sensor Data}
\begin{changemargin}{1cm}
This lab is about interpreting sensor data.

In particular, you are going to
read the accelerometer data and count the number of steps taken by the holder
of the phone. 

There are two main problems:\\
\quad 1) sensor data is noisy; and\\
\quad 2) you need
to recognize when a step occurs.
\end{changemargin}
\end{frame}

\begin{frame}[fragile]
\frametitle{Sensor Noise}
\begin{changemargin}{1cm}


To deal with sensor noise, smooth the data.

Below is a very simple low-pass filter:

\begin{verbatim}
  smoothedAccel += (newValue - smoothedAccel) / C;
\end{verbatim}

\end{changemargin}
\end{frame}

\begin{frame}
\frametitle{Recognizing a Step}
\begin{changemargin}{1cm}

To recognize a step, you'll need to identify a change in the value of the $y$-axis
acceleration. 

Identifying a change means that you'll need the previous value along
with the current value. 

You could do that using a finite state machine (as described
in the lab manual). 

You could do that by doing tests both on the previous value 
and the current value. 

Can you think of any pattern that might describe a step?

\end{changemargin}
\end{frame}


\part{Toast, Broadcast Receivers, \& Lists}
\frame{\partpage}

\begin{frame}[fragile]
\frametitle{Toast!}
\begin{changemargin}{1cm}

 Sometimes you want to display a short message to the user.
Use {\tt Toast} to do that. Just include the following code:
\begin{verbatim}
  Toast.makeText(getApplicationContext(), 
                 "A Toast!", 
                 Toast.LENGTH_LONG).show();
\end{verbatim}
It's also OK to use in your onClick event listener:
\begin{verbatim}
  Toast.makeText(MainActivity.this, "A Toast!", 
                 Toast.LENGTH_LONG).show();
\end{verbatim}


\end{changemargin}
\end{frame}


\begin{frame}
\frametitle{Broadcast Receivers}
\begin{changemargin}{1cm}

We talked earlier about the concept of Android Intents.

Android also uses Intents to broadcast information about what's
happening on the system between applications. 

Applications want to
know about events such as the phone being plugged into a power source;
or, screen rotation.


\end{changemargin}
\end{frame}

\begin{frame}
\frametitle{Broadcast Analogy}
\begin{changemargin}{1cm}

A ``party line''
for Android Broadcast Receivers. 

Many different applications can
register an event listener, and they all get notified whenever
something happens.


\end{changemargin}
\end{frame}


\begin{frame}[fragile]
\frametitle{Broadcast Receiver Example: Rotation Listener}
\begin{changemargin}{0.5cm}
{\scriptsize
\begin{verbatim}
 BroadcastReceiver broadcastReceiver = new BroadcastReceiver() {
    @Override
    public void onReceive(Context c, Intent i) {
      int orientation = c.getResources().getConfiguration().orientation;
      if (orientation == Configuration.ORIENTATION_PORTRAIT) {
        Toast.makeText(c, "Portrait", Toast.LENGTH_SHORT).show();
      } else if (orientation == Configuration.ORIENTATION_LANDSCAPE) {
        Toast.makeText(c, "Landscape", Toast.LENGTH_SHORT).show();
      } else {
        Toast.makeText(c, "???", Toast.LENGTH_SHORT).show();
      }
    }
  };
\end{verbatim}

}
\end{changemargin}
\end{frame}

\begin{frame}
\frametitle{Register \& Unregister the Receiver}
\begin{changemargin}{1cm}
To register: {\scriptsize \texttt{registerReceiver(broadcastReceiver, intentFilter);}}

To unregister: {\scriptsize \texttt{unregisterReceiver(broadcastReceiver);}}

Unfortunately, by default Android destroys your app on a screen rotate event. 

This example only works if you tell Android not to do that.
\end{changemargin}
\end{frame}

\begin{frame}[fragile]
\frametitle{Broadcast Receiver: Phone Ring Listener}
\begin{changemargin}{1cm}

This time to listen for phone calls. 

First, you need to modify the {\tt manifest.xml} file to permit
the app to listen for phone calls:



{\scriptsize
\begin{verbatim}
<uses-permission android:name="android.permission.READ_PHONE_STATE">
</uses-permission>
\end{verbatim}
}
\end{changemargin}


\end{frame}


\begin{frame}[fragile]
\frametitle{Registering it in the Manifest}
\begin{changemargin}{1cm}

Include inside the \verb+<application>+ tag:
{\scriptsize
\begin{verbatim}
<receiver android:name="ca.patricklam.ece155demo.MyPhoneReceiver" >
    <intent-filter>
        <action android:name="android.intent.action.PHONE_STATE" />
    </intent-filter>
</receiver>
\end{verbatim}
}
\end{changemargin}
\end{frame}

\begin{frame}[fragile]
\frametitle{Phone Receiver}
\begin{changemargin}{1cm}
This means that we have to create a separate class {\tt MyPhoneReceiver}:
{\scriptsize
\begin{verbatim}
public class MyPhoneReceiver extends BroadcastReceiver {
  @Override
  public void onReceive(Context context, Intent intent) {
    Bundle extras = intent.getExtras();
    if (extras != null) {
      String state = extras.getString(TelephonyManager.EXTRA_STATE);
      Log.w("PHONE", state);
      if (state.equals(TelephonyManager.EXTRA_STATE_RINGING)) {
        String phoneNumber = extras.getString
                               (TelephonyManager.EXTRA_INCOMING_NUMBER);
        Log.w("PHONE", phoneNumber);
      }
    }
  }
}
\end{verbatim}

}


\end{changemargin}
\end{frame}

\begin{frame}
\frametitle{Lists}
\begin{changemargin}{1cm}

Another useful UI element is the {\tt ListView}.

We can use it to show 
a list of items to the user (for instance, so that the user can choose
one of the list elements). 

There are a couple of caveats with using the
{\tt ListView}.

\end{changemargin}
\end{frame}

\begin{frame}[fragile]
\frametitle{Creating a ListView}
\begin{changemargin}{1cm}

First, create a {\tt ListView} object by dragging it onto the Activity's XML file.
\begin{itemize}
\item Note: you have to manually edit the {\tt android:id} attribute so
that its value is \verb+@android:id/list+.
\end{itemize}
Next, change your Activity to be a ListActivity. 

Add a field {\tt listAdapter} of type \verb+ArrayAdapter<String>+.


\end{changemargin}
\end{frame}

\begin{frame}[fragile]
\frametitle{Populating a ListView}
\begin{changemargin}{1cm}

Populate the list with entries:

{\scriptsize
\begin{verbatim}
  List<String> data = new ArrayList<String>();
  data.add("ECE155");
  data.add("ECE106");
  data.add("ECE124");
  listAdapter = new ArrayAdapter<String>(this,
                           android.R.layout.simple_list_item_1,
                           data);
  setListAdapter(listAdapter);
\end{verbatim}
}
\end{changemargin}
\end{frame}


\begin{frame}[fragile]
\frametitle{Clicking an Item}
\begin{changemargin}{1cm}

Finally, we want something to happen when we click on a list item:

{\scriptsize
\begin{verbatim}
  @Override
  protected void onListItemClick(ListView l, View v, 
                                    int position, long id) {
    super.onListItemClick(l, v, position, id);
    String s = (String) getListAdapter().getItem(position);
    Toast.makeText(this, "Aha: "+s, Toast.LENGTH_SHORT).show();
  }
\end{verbatim}
}
Display a toast when the user chooses a list item.


\end{changemargin}
\end{frame}


\begin{frame}[fragile]
\frametitle{Dynamically Updating a ListView}
\begin{changemargin}{1cm}
{\scriptsize
Of course, we can also
add items to the {\tt ListView} programmatically. In a click listener (or
anywhere else), you can write:
\begin{verbatim}
  String now = String.valueOf(System.currentTimeMillis());
  listAdapter.add(now);
\end{verbatim}

}
\end{changemargin}
\end{frame}


\begin{frame}
\frametitle{ListView}
\begin{changemargin}{1cm}

Note that adding elements to the ArrayAdapter also adds them to the ListActivity.

Also, we currently store the ArrayAdapter as a field.


That means it'll go away whenever the Activity is destroyed (e.g.
rotation). 

We'll want to do something about that.

\end{changemargin}
\end{frame}

\end{document}
