
\documentclass[letterpaper,hide notes,xcolor={table,svgnames},pdftex]{beamer}
\def\showexamples{t}


%\usepackage[svgnames]{xcolor}

%% Demo talk
%\documentclass[letterpaper,notes=show]{beamer}

\usecolortheme{crane}
\setbeamertemplate{navigation symbols}{}

\usetheme{MyPittsburgh}
%\usetheme{Frankfurt}

%\usepackage{tipa}

\usepackage{hyperref}
\usepackage{graphicx,xspace}
\usepackage[normalem]{ulem}

\newcommand\SF[1]{$\bigstar$\footnote{SF: #1}}



\newcounter{tmpnumSlide}
\newcounter{tmpnumNote}

% old question code
%\newcommand\question[1]{{$\bigstar$ \small \onlySlide{2}{#1}}}
% \newcommand\nquestion[1]{\ifdefined \presentationonly \textcircled{?} \fi \note{\par{\Large \textbf{?}} #1}}
% \newcommand\nanswer[1]{\note{\par{\Large \textbf{A}} #1}}


 \newcommand\mnote[1]{%
   \addtocounter{tmpnumSlide}{1}
   \ifdefined\showcues {~\tiny\fbox{\arabic{tmpnumSlide}}}\fi
   \note{\setlength{\parskip}{1ex}\addtocounter{tmpnumNote}{1}\textbf{\Large \arabic{tmpnumNote}:} {#1\par}}}

\newcommand\mmnote[1]{\note{\setlength{\parskip}{1ex}#1\par}}

%\newcommand\mnote[2][]{\ifdefined\handoutwithnotes {~\tiny\fbox{#1}}\fi
% \note{\setlength{\parskip}{1ex}\textbf{\Large #1:} #2\par}}

%\newcommand\mnote[2][]{{\tiny\fbox{#1}} \note{\setlength{\parskip}{1ex}\textbf{\Large #1:} #2\par}}

\newcommand\mquestion[2]{{~\color{red}\fbox{?}}\note{\setlength{\parskip}{1ex}\par{\Large \textbf{?}} #1} \note{\setlength{\parskip}{1ex}\par{\Large \textbf{A}} #2\par}\ifdefined \presentationonly \pause \fi}

\newcommand\blackboard[1]{%
\ifdefined   \showblackboard
  {#1}
  \else {\begin{center} \fbox{\colorbox{blue!30}{%
         \begin{minipage}{.95\linewidth}%
           \hspace{\stretch{1}} Some space intentionally left blank; done at the blackboard.%
         \end{minipage}}}\end{center}}%
         \fi%
}



%\newcommand\q{\tikz \node[thick,color=black,shape=circle]{?};}
%\newcommand\q{\ifdefined \presentationonly \textcircled{?} \fi}

\usepackage{listings}
\lstset{%
  keywordstyle=\bfseries,
  aboveskip=15pt,
  belowskip=15pt,
  captionpos=b,
  identifierstyle=\ttfamily,
  escapeinside={(*@}{@*)},
  stringstyle=\ttfamiliy,
  frame=lines,
  numbers=left, basicstyle=\scriptsize, numberstyle=\tiny, stepnumber=0, numbersep=2pt}

\usepackage{siunitx}
\newcommand\sius[1]{\num[group-separator = {,}]{#1}\si{\micro\second}}
\newcommand\sims[1]{\num[group-separator = {,}]{#1}\si{\milli\second}}
\newcommand\sins[1]{\num[group-separator = {,}]{#1}\si{\nano\second}}
\sisetup{group-separator = {,}, group-digits = true}

%% -------------------- tikz --------------------
\usepackage{tikz}
\usetikzlibrary{positioning}
\usetikzlibrary{arrows,backgrounds,automata,decorations.shapes,decorations.pathmorphing,decorations.markings,decorations.text}

\tikzstyle{place}=[circle,draw=blue!50,fill=blue!20,thick, inner sep=0pt,minimum size=6mm]
\tikzstyle{transition}=[rectangle,draw=black!50,fill=black!20,thick, inner sep=0pt,minimum size=4mm]

\tikzstyle{block}=[rectangle,draw=black, thick, inner sep=5pt]
\tikzstyle{bullet}=[circle,draw=black, fill=black, thin, inner sep=2pt]

\tikzstyle{pre}=[<-,shorten <=1pt,>=stealth',semithick]
\tikzstyle{post}=[->,shorten >=1pt,>=stealth',semithick]
\tikzstyle{bi}=[<->,shorten >=1pt,shorten <=1pt, >=stealth',semithick]

\tikzstyle{mut}=[-,>=stealth',semithick]

\tikzstyle{treereset}=[dashed,->, shorten >=1pt,>=stealth',thin]

\usepackage{ifmtarg}
\usepackage{xifthen}
\makeatletter
% new counter to now which frame it is within the sequence
\newcounter{multiframecounter}
% initialize buffer for previously used frame title
\gdef\lastframetitle{\textit{undefined}}
% new environment for a multi-frame
\newenvironment{multiframe}[1][]{%
\ifthenelse{\isempty{#1}}{%
% if no frame title was set via optional parameter,
% only increase sequence counter by 1
\addtocounter{multiframecounter}{1}%
}{%
% new frame title has been provided, thus
% reset sequence counter to 1 and buffer frame title for later use
\setcounter{multiframecounter}{1}%
\gdef\lastframetitle{#1}%
}%
% start conventional frame environment and
% automatically set frame title followed by sequence counter
\begin{frame}%
\frametitle{\lastframetitle~{\normalfont(\arabic{multiframecounter})}}%
}{%
\end{frame}%
}
\makeatother

\makeatletter
\newdimen\tu@tmpa%
\newdimen\ydiffl%
\newdimen\xdiffl%
\newcommand\ydiff[2]{%
    \coordinate (tmpnamea) at (#1);%
    \coordinate (tmpnameb) at (#2);%
    \pgfextracty{\tu@tmpa}{\pgfpointanchor{tmpnamea}{center}}%
    \pgfextracty{\ydiffl}{\pgfpointanchor{tmpnameb}{center}}%
    \advance\ydiffl by -\tu@tmpa%
}
\newcommand\xdiff[2]{%
    \coordinate (tmpnamea) at (#1);%
    \coordinate (tmpnameb) at (#2);%
    \pgfextractx{\tu@tmpa}{\pgfpointanchor{tmpnamea}{center}}%
    \pgfextractx{\xdiffl}{\pgfpointanchor{tmpnameb}{center}}%
    \advance\xdiffl by -\tu@tmpa%
}
\makeatother
\newcommand{\copyrightbox}[3][r]{%
\begin{tikzpicture}%
\node[inner sep=0pt,minimum size=2em](ciimage){#2};
\usefont{OT1}{phv}{n}{n}\fontsize{4}{4}\selectfont
\ydiff{ciimage.south}{ciimage.north}
\xdiff{ciimage.west}{ciimage.east}
\ifthenelse{\equal{#1}{r}}{%
\node[inner sep=0pt,right=1ex of ciimage.south east,anchor=north west,rotate=90]%
{\raggedleft\color{black!50}\parbox{\the\ydiffl}{\raggedright{}#3}};%
}{%
\ifthenelse{\equal{#1}{l}}{%
\node[inner sep=0pt,right=1ex of ciimage.south west,anchor=south west,rotate=90]%
{\raggedleft\color{black!50}\parbox{\the\ydiffl}{\raggedright{}#3}};%
}{%
\node[inner sep=0pt,below=1ex of ciimage.south west,anchor=north west]%
{\raggedleft\color{black!50}\parbox{\the\xdiffl}{\raggedright{}#3}};%
}
}
\end{tikzpicture}
}


%% --------------------

%\usepackage[excludeor]{everyhook}
%\PushPreHook{par}{\setbox0=\lastbox\llap{MUH}}\box0}

%\vspace*{\stretch{1}

%\setbox0=\lastbox \llap{\textbullet\enskip}\box0}

\setlength{\parskip}{\fill}

\newcommand\noskips{\setlength{\parskip}{1ex}}
\newcommand\doskips{\setlength{\parskip}{\fill}}

\newcommand\xx{\par\vspace*{\stretch{1}}\par}
\newcommand\xxs{\par\vspace*{2ex}\par}
\newcommand\tuple[1]{\langle #1 \rangle}
\newcommand\code[1]{{\sf \footnotesize #1}}
\newcommand\ex[1]{\uline{Example:} \ifdefined \presentationonly \pause \fi
  \ifdefined\showexamples#1\xspace\else{\uline{\hspace*{2cm}}}\fi}

\newcommand\ceil[1]{\lceil #1 \rceil}


\AtBeginSection[]
{
   \begin{frame}
       \frametitle{Outline}
       \tableofcontents[currentsection]
   \end{frame}
}



\pgfdeclarelayer{edgelayer}
\pgfdeclarelayer{nodelayer}
\pgfsetlayers{edgelayer,nodelayer,main}

\tikzstyle{none}=[inner sep=0pt]
\tikzstyle{rn}=[circle,fill=Red,draw=Black,line width=0.8 pt]
\tikzstyle{gn}=[circle,fill=Lime,draw=Black,line width=0.8 pt]
\tikzstyle{yn}=[circle,fill=Yellow,draw=Black,line width=0.8 pt]
\tikzstyle{empty}=[circle,fill=White,draw=Black]
\tikzstyle{bw} = [rectangle, draw, fill=blue!20, 
    text width=4em, text centered, rounded corners, minimum height=2em]
    
    \newcommand{\CcNote}[1]{% longname
	This work is licensed under the \textit{Creative Commons #1 3.0 License}.%
}
\newcommand{\CcImageBy}[1]{%
	\includegraphics[scale=#1]{creative_commons/cc_by_30.pdf}%
}
\newcommand{\CcImageSa}[1]{%
	\includegraphics[scale=#1]{creative_commons/cc_sa_30.pdf}%
}
\newcommand{\CcImageNc}[1]{%
	\includegraphics[scale=#1]{creative_commons/cc_nc_30.pdf}%
}
\newcommand{\CcGroupBySa}[2]{% zoom, gap
	\CcImageBy{#1}\hspace*{#2}\CcImageNc{#1}\hspace*{#2}\CcImageSa{#1}%
}
\newcommand{\CcLongnameByNcSa}{Attribution-NonCommercial-ShareAlike}

\newenvironment{changemargin}[1]{% 
  \begin{list}{}{% 
    \setlength{\topsep}{0pt}% 
    \setlength{\leftmargin}{#1}% 
    \setlength{\rightmargin}{1em}
    \setlength{\listparindent}{\parindent}% 
    \setlength{\itemindent}{\parindent}% 
    \setlength{\parsep}{\parskip}% 
  }% 
  \item[]}{\end{list}} 




\usepackage{alltt}

\title{Lecture 2 --- Introduction to Java}

\author{Patrick Lam \& Jeff Zarnett \\ \small \texttt{p.lam@ece.uwaterloo.ca} \& \texttt{jzarnett@uwaterloo.ca}}
\institute{Department of Electrical and Computer Engineering \\
  University of Waterloo}
\date{\today}


\begin{document}


\begin{frame}
  \titlepage
\end{frame}

\begin{frame}
\frametitle{Java for C\# Programmers}
\begin{changemargin}{1cm}
The labs this semester will require you to write Java
code for the Android platform, yet you learned C\# in ECE~150.


Fortunately, there are a lot of similarities between Java and C\#,
so you should have a smooth transition. 

As it says in the syllabus, please take some time at the beginning of the term to get caught up on Java. 


\end{changemargin}
\end{frame}

\begin{frame}
\frametitle{Java: Object Oriented}
\begin{changemargin}{1cm}

Java is an object-oriented programming language: 
\begin{itemize}
\item Every piece of data is encapsulated in some object.
\item Every executable statement is in some method.
\item Every object is an instance of a class (or is an array).
\end{itemize}

\end{changemargin}
\end{frame}

\begin{frame}
\frametitle{Java Types}
\begin{changemargin}{1cm}

Java has eight primitive (basic) types. Every variable will be one of the primitive types or a reference to a Java object.

A reference may be \texttt{null} or contain the address of an instance of an object. 

\end{changemargin}
\end{frame}

\begin{frame}
\frametitle{The Eight Basic Types}
\begin{changemargin}{1cm}

The eight basic types are:

\begin{enumerate}
\item \texttt{boolean}
\item \texttt{char}
\item \texttt{byte}
\item \texttt{short}
\item \texttt{int}
\item \texttt{long}
\item \texttt{float}
\item \texttt{double}
\end{enumerate}


\end{changemargin}
\end{frame}

\begin{frame}
\frametitle{Floating Point Special Values}
\begin{changemargin}{1cm}

Note that in addition to their normal values, the floating point types have some extra weird values: 

\texttt{NEGATIVE\_INFINITY}, \texttt{POSITIVE\_INFINITY}, and \texttt{NaN}.

These special values result from operations that go out of range or make no mathematical sense.

\end{changemargin}
\end{frame}

\begin{frame}
\frametitle{Java Types}
\begin{changemargin}{1cm}

he most common types you are likely to use are \texttt{boolean}, \texttt{int}, and \texttt{double} (in the labs, \texttt{float} gets a fair amount of use).

You have no doubt noticed that \texttt{String} does not appear.

It's not a simple type; it's a reference to an array of characters.

\end{changemargin}
\end{frame}

\begin{frame}
\frametitle{Strings}
\begin{changemargin}{1cm}

Declare a \texttt{String} as \texttt{null}, like this: \texttt{String s = null;}.

Strings are immutable: once created, they don't change. 

If you add to a String or replace characters in it, you get a new String back and not the old one. 

Example: \texttt{string1.replaceAll('?', '.')}, $\rightarrow$ assign that result  (\texttt{string1 = string1.replaceAll('?', '.')}). 

\end{changemargin}
\end{frame}


\begin{frame}
\frametitle{Strings}
\begin{changemargin}{1cm}

The equality operator \texttt{==} can behave strangely on Strings. 

Use the \texttt{equals} method:\\
\quad \texttt{if (string1.equals(string2) \{ \ldots~\} }.

\end{changemargin}
\end{frame}


\begin{frame}
\frametitle{Wrapper Classes}
\begin{changemargin}{1cm}
Java also provides wrapper classes for the primitive types.

Work with them as if they were regular objects. 

The wrapper class for \texttt{int}: \texttt{Integer} 

(Mostly, just capitalize the first letter). 
\end{changemargin}
\end{frame}

\begin{frame}
\frametitle{Complex Objects}
\begin{changemargin}{1cm}
Just like in C\# you create them in classes. 

To create an instance of that class you use the \texttt{new} keyword. 

Example: \texttt{Integer example = new Integer(20);}. 

The \texttt{new} keyword invokes the class Constructor.
\end{changemargin}
\end{frame}

\begin{frame}
\frametitle{Java Semantics}
\begin{changemargin}{1cm}

Java does not have \texttt{struct}s; classes only. 

Frequently asked questions: there are no pointers or delegates.

Java uses \textit{Garbage Collection}. No freeing up or de-allocating objects that are no longer needed.

This prevents the extremely common error of trying to use some object/memory that has been released.


\end{changemargin}
\end{frame}

\begin{frame}
\frametitle{Imperative Constructs}
\begin{changemargin}{1cm}

\begin{itemize}
\item Assignment: {\tt x = y;}
\item Math: {\tt i = j + k}. (This includes the operators like += etc.)
\item Expressions: {\tt z > 10 || ((c == 0) \&\& (a == b))}
\item If-Statements: {\tt if(cond) \{ \ldots~\} else if (cond2) \{ \ldots~\}  else  \{ \ldots~\} }
\item For Loops: {\tt for (init; cond; expr2) \{ \ldots~\} }
\item While Loops: {\tt while (cond) \{ \ldots~\} }
\item Do-While Loops: {\tt do \{ \ldots~\} while (cond); }
\item Switch-Case Statements: {\tt switch (v) \{ case N: \ldots; break; case M: \ldots; break; default: \ldots; break; \} }
\end{itemize}
However, C\#'s {\tt foreach (type t in c)} is instead {\tt for (Type t : c)} in Java.

\end{changemargin}
\end{frame}

\begin{frame}[fragile]
\frametitle{Methods}
\begin{changemargin}{1cm}

Methods work the same way also:
\begin{verbatim}
modifiers returnType methodName(param-list) {
  T1 t; returnType r;
  ...
  return r;
}
\end{verbatim}

When you call a method on an object, you still use the \texttt{.} between the object and the method, such as \texttt{t.toString()}.

In C\#, the method name convention is
{\tt UppercaseFirstLetter()}, while in Java, it is {\tt lowercaseFirstLetter()}.

\end{changemargin}
\end{frame}


\begin{frame}[fragile]
\frametitle{Simple Unit Converter: C\#}
\begin{changemargin}{1cm}

{\scriptsize
\begin{verbatim}
using System;

class FootConverter {
  static double ConvertFeetToMeters(double feet) {
    return feet * 3.28;
  }
  static void Main(string[] s) {
    Console.WriteLine("{0} ft is {1} m.", s[0], 
      ConvertFeetToMeters(Convert.ToDouble(s[0])));
  }
}
\end{verbatim}
}

\end{changemargin}
\end{frame}

\begin{frame}[fragile]
\frametitle{Simple Unit Converter: Java}
\begin{changemargin}{1cm}

{\scriptsize
\begin{verbatim}
class FootConverter {
  static double convertFeetToMeters(double feet) {
    return feet * 3.28;
  }
  public static void main(String[] s) {
    System.out.printf("%s ft is %.2f m.\n", s[0],
      convertFeetToMeters(Double.parseDouble(s[0])));
  }
}
\end{verbatim}
}

\end{changemargin}
\end{frame}

\begin{frame}
\frametitle{Arrays \& Collections}
\begin{changemargin}{1cm}

The simple array is created with the \texttt{[]} square brackets. 

Example: \texttt{int[] numbers = new int[10];}

You can have null elements in an array (say, of Strings) without this affecting the array length. 

An array is technically an object so you can assign it where a generic \texttt{Object} is expected. 


\end{changemargin}
\end{frame}

\begin{frame}
\frametitle{Arrays}
\begin{changemargin}{1cm}

Multidimensional arrays are also allowed, but only for primitive types \texttt{int[][][] coordinates = new int[5][10][2];}.  

This is not a big restriction because you can just have an array of arrays. 

Unlike C\# though, you can't specify a rectangular array.

\end{changemargin}
\end{frame}

\begin{frame}
\frametitle{Arrays}
\begin{changemargin}{1cm}

This is great and all, but like the String the explicit array is of fixed size.

Create a new, bigger one if you needed it and copy all the data to the bigger one...? 

Allocate an array of size 999 when we aren't sure how many we'll need?

Wouldn't it be nice if we had a dynamic collection?


\end{changemargin}
\end{frame}


\begin{frame}
\frametitle{Java Collections}
\begin{changemargin}{1cm}

In Java, we do, and they're called, \texttt{Collection}s. 

The most common one: the \texttt{List}. 

The type List takes a parameter in angle brackets to tell you what this is a list of. 

Example: \texttt{List<String>}.

\end{changemargin}
\end{frame}

\begin{frame}
\frametitle{Lists}
\begin{changemargin}{1cm}

Note that you can't call \texttt{new List<String>()} because no constructor exists for just plain \texttt{List}. 

Be specific about what kind of list you want to have, such as \texttt{ArrayList} (a very common one) or \texttt{LinkedList}.


\end{changemargin}
\end{frame}

\begin{frame}
\frametitle{Three Basic Collections}
\begin{changemargin}{1cm}

Three basic collections exist:

\begin{enumerate}
	\item List
	\item Map (in a later lecture)
	\item Set (like a list, but no duplicates)
\end{enumerate}


\end{changemargin}
\end{frame}

\begin{frame}[fragile]
\frametitle{Your First Java Program: C\# Equivalent}
\begin{changemargin}{1cm}

\begin{verbatim}
using System;

class C {
  static void Main() {
    Console.WriteLine("Hello, world!");
  }
}
\end{verbatim}


\end{changemargin}
\end{frame}

\begin{frame}[fragile]
\frametitle{Your First Java Program}
\begin{changemargin}{1cm}

\begin{verbatim}
class C {
  public static void main(String[] argv) {s
    System.out.println("Hello, world!");
  }
}
\end{verbatim}


\end{changemargin}
\end{frame}


\begin{frame}
\frametitle{Logging for Android}
\begin{changemargin}{1cm}
Android tip: {\tt System.out.println()} is great
for debugging console applications, but doesn't work on Android.

{\tt ~~~Log.d("tag", "i = "+i); }

This writes out a debug (d) logging message, which appears e.g. in your Eclipse
{\tt LogCat} window.


You can then filter out logging messages
by level or tag, so that you only see the ones you're interested in.

\end{changemargin}
\end{frame}


\begin{frame}
\frametitle{Eclipse Demo}
\begin{changemargin}{1cm}

Now we'll open up Eclipse and create some basic Java programs. We can take a look at the syntax of Java in simple situations.


\end{changemargin}
\end{frame}



\end{document}

