
\documentclass[letterpaper,hide notes,xcolor={table,svgnames},pdftex]{beamer}
\def\showexamples{t}


%\usepackage[svgnames]{xcolor}

%% Demo talk
%\documentclass[letterpaper,notes=show]{beamer}

\usecolortheme{crane}
\setbeamertemplate{navigation symbols}{}

\usetheme{MyPittsburgh}
%\usetheme{Frankfurt}

%\usepackage{tipa}

\usepackage{hyperref}
\usepackage{graphicx,xspace}
\usepackage[normalem]{ulem}

\newcommand\SF[1]{$\bigstar$\footnote{SF: #1}}



\newcounter{tmpnumSlide}
\newcounter{tmpnumNote}

% old question code
%\newcommand\question[1]{{$\bigstar$ \small \onlySlide{2}{#1}}}
% \newcommand\nquestion[1]{\ifdefined \presentationonly \textcircled{?} \fi \note{\par{\Large \textbf{?}} #1}}
% \newcommand\nanswer[1]{\note{\par{\Large \textbf{A}} #1}}


 \newcommand\mnote[1]{%
   \addtocounter{tmpnumSlide}{1}
   \ifdefined\showcues {~\tiny\fbox{\arabic{tmpnumSlide}}}\fi
   \note{\setlength{\parskip}{1ex}\addtocounter{tmpnumNote}{1}\textbf{\Large \arabic{tmpnumNote}:} {#1\par}}}

\newcommand\mmnote[1]{\note{\setlength{\parskip}{1ex}#1\par}}

%\newcommand\mnote[2][]{\ifdefined\handoutwithnotes {~\tiny\fbox{#1}}\fi
% \note{\setlength{\parskip}{1ex}\textbf{\Large #1:} #2\par}}

%\newcommand\mnote[2][]{{\tiny\fbox{#1}} \note{\setlength{\parskip}{1ex}\textbf{\Large #1:} #2\par}}

\newcommand\mquestion[2]{{~\color{red}\fbox{?}}\note{\setlength{\parskip}{1ex}\par{\Large \textbf{?}} #1} \note{\setlength{\parskip}{1ex}\par{\Large \textbf{A}} #2\par}\ifdefined \presentationonly \pause \fi}

\newcommand\blackboard[1]{%
\ifdefined   \showblackboard
  {#1}
  \else {\begin{center} \fbox{\colorbox{blue!30}{%
         \begin{minipage}{.95\linewidth}%
           \hspace{\stretch{1}} Some space intentionally left blank; done at the blackboard.%
         \end{minipage}}}\end{center}}%
         \fi%
}



%\newcommand\q{\tikz \node[thick,color=black,shape=circle]{?};}
%\newcommand\q{\ifdefined \presentationonly \textcircled{?} \fi}

\usepackage{listings}
\lstset{%
  keywordstyle=\bfseries,
  aboveskip=15pt,
  belowskip=15pt,
  captionpos=b,
  identifierstyle=\ttfamily,
  escapeinside={(*@}{@*)},
  stringstyle=\ttfamiliy,
  frame=lines,
  numbers=left, basicstyle=\scriptsize, numberstyle=\tiny, stepnumber=0, numbersep=2pt}

\usepackage{siunitx}
\newcommand\sius[1]{\num[group-separator = {,}]{#1}\si{\micro\second}}
\newcommand\sims[1]{\num[group-separator = {,}]{#1}\si{\milli\second}}
\newcommand\sins[1]{\num[group-separator = {,}]{#1}\si{\nano\second}}
\sisetup{group-separator = {,}, group-digits = true}

%% -------------------- tikz --------------------
\usepackage{tikz}
\usetikzlibrary{positioning}
\usetikzlibrary{arrows,backgrounds,automata,decorations.shapes,decorations.pathmorphing,decorations.markings,decorations.text}

\tikzstyle{place}=[circle,draw=blue!50,fill=blue!20,thick, inner sep=0pt,minimum size=6mm]
\tikzstyle{transition}=[rectangle,draw=black!50,fill=black!20,thick, inner sep=0pt,minimum size=4mm]

\tikzstyle{block}=[rectangle,draw=black, thick, inner sep=5pt]
\tikzstyle{bullet}=[circle,draw=black, fill=black, thin, inner sep=2pt]

\tikzstyle{pre}=[<-,shorten <=1pt,>=stealth',semithick]
\tikzstyle{post}=[->,shorten >=1pt,>=stealth',semithick]
\tikzstyle{bi}=[<->,shorten >=1pt,shorten <=1pt, >=stealth',semithick]

\tikzstyle{mut}=[-,>=stealth',semithick]

\tikzstyle{treereset}=[dashed,->, shorten >=1pt,>=stealth',thin]

\usepackage{ifmtarg}
\usepackage{xifthen}
\makeatletter
% new counter to now which frame it is within the sequence
\newcounter{multiframecounter}
% initialize buffer for previously used frame title
\gdef\lastframetitle{\textit{undefined}}
% new environment for a multi-frame
\newenvironment{multiframe}[1][]{%
\ifthenelse{\isempty{#1}}{%
% if no frame title was set via optional parameter,
% only increase sequence counter by 1
\addtocounter{multiframecounter}{1}%
}{%
% new frame title has been provided, thus
% reset sequence counter to 1 and buffer frame title for later use
\setcounter{multiframecounter}{1}%
\gdef\lastframetitle{#1}%
}%
% start conventional frame environment and
% automatically set frame title followed by sequence counter
\begin{frame}%
\frametitle{\lastframetitle~{\normalfont(\arabic{multiframecounter})}}%
}{%
\end{frame}%
}
\makeatother

\makeatletter
\newdimen\tu@tmpa%
\newdimen\ydiffl%
\newdimen\xdiffl%
\newcommand\ydiff[2]{%
    \coordinate (tmpnamea) at (#1);%
    \coordinate (tmpnameb) at (#2);%
    \pgfextracty{\tu@tmpa}{\pgfpointanchor{tmpnamea}{center}}%
    \pgfextracty{\ydiffl}{\pgfpointanchor{tmpnameb}{center}}%
    \advance\ydiffl by -\tu@tmpa%
}
\newcommand\xdiff[2]{%
    \coordinate (tmpnamea) at (#1);%
    \coordinate (tmpnameb) at (#2);%
    \pgfextractx{\tu@tmpa}{\pgfpointanchor{tmpnamea}{center}}%
    \pgfextractx{\xdiffl}{\pgfpointanchor{tmpnameb}{center}}%
    \advance\xdiffl by -\tu@tmpa%
}
\makeatother
\newcommand{\copyrightbox}[3][r]{%
\begin{tikzpicture}%
\node[inner sep=0pt,minimum size=2em](ciimage){#2};
\usefont{OT1}{phv}{n}{n}\fontsize{4}{4}\selectfont
\ydiff{ciimage.south}{ciimage.north}
\xdiff{ciimage.west}{ciimage.east}
\ifthenelse{\equal{#1}{r}}{%
\node[inner sep=0pt,right=1ex of ciimage.south east,anchor=north west,rotate=90]%
{\raggedleft\color{black!50}\parbox{\the\ydiffl}{\raggedright{}#3}};%
}{%
\ifthenelse{\equal{#1}{l}}{%
\node[inner sep=0pt,right=1ex of ciimage.south west,anchor=south west,rotate=90]%
{\raggedleft\color{black!50}\parbox{\the\ydiffl}{\raggedright{}#3}};%
}{%
\node[inner sep=0pt,below=1ex of ciimage.south west,anchor=north west]%
{\raggedleft\color{black!50}\parbox{\the\xdiffl}{\raggedright{}#3}};%
}
}
\end{tikzpicture}
}


%% --------------------

%\usepackage[excludeor]{everyhook}
%\PushPreHook{par}{\setbox0=\lastbox\llap{MUH}}\box0}

%\vspace*{\stretch{1}

%\setbox0=\lastbox \llap{\textbullet\enskip}\box0}

\setlength{\parskip}{\fill}

\newcommand\noskips{\setlength{\parskip}{1ex}}
\newcommand\doskips{\setlength{\parskip}{\fill}}

\newcommand\xx{\par\vspace*{\stretch{1}}\par}
\newcommand\xxs{\par\vspace*{2ex}\par}
\newcommand\tuple[1]{\langle #1 \rangle}
\newcommand\code[1]{{\sf \footnotesize #1}}
\newcommand\ex[1]{\uline{Example:} \ifdefined \presentationonly \pause \fi
  \ifdefined\showexamples#1\xspace\else{\uline{\hspace*{2cm}}}\fi}

\newcommand\ceil[1]{\lceil #1 \rceil}


\AtBeginSection[]
{
   \begin{frame}
       \frametitle{Outline}
       \tableofcontents[currentsection]
   \end{frame}
}



\pgfdeclarelayer{edgelayer}
\pgfdeclarelayer{nodelayer}
\pgfsetlayers{edgelayer,nodelayer,main}

\tikzstyle{none}=[inner sep=0pt]
\tikzstyle{rn}=[circle,fill=Red,draw=Black,line width=0.8 pt]
\tikzstyle{gn}=[circle,fill=Lime,draw=Black,line width=0.8 pt]
\tikzstyle{yn}=[circle,fill=Yellow,draw=Black,line width=0.8 pt]
\tikzstyle{empty}=[circle,fill=White,draw=Black]
\tikzstyle{bw} = [rectangle, draw, fill=blue!20, 
    text width=4em, text centered, rounded corners, minimum height=2em]
    
    \newcommand{\CcNote}[1]{% longname
	This work is licensed under the \textit{Creative Commons #1 3.0 License}.%
}
\newcommand{\CcImageBy}[1]{%
	\includegraphics[scale=#1]{creative_commons/cc_by_30.pdf}%
}
\newcommand{\CcImageSa}[1]{%
	\includegraphics[scale=#1]{creative_commons/cc_sa_30.pdf}%
}
\newcommand{\CcImageNc}[1]{%
	\includegraphics[scale=#1]{creative_commons/cc_nc_30.pdf}%
}
\newcommand{\CcGroupBySa}[2]{% zoom, gap
	\CcImageBy{#1}\hspace*{#2}\CcImageNc{#1}\hspace*{#2}\CcImageSa{#1}%
}
\newcommand{\CcLongnameByNcSa}{Attribution-NonCommercial-ShareAlike}

\newenvironment{changemargin}[1]{% 
  \begin{list}{}{% 
    \setlength{\topsep}{0pt}% 
    \setlength{\leftmargin}{#1}% 
    \setlength{\rightmargin}{1em}
    \setlength{\listparindent}{\parindent}% 
    \setlength{\itemindent}{\parindent}% 
    \setlength{\parsep}{\parskip}% 
  }% 
  \item[]}{\end{list}} 




\title{Lecture 8 --- Software Design Patterns}

\author{Jeff Zarnett \\ \small \texttt{jzarnett@uwaterloo.ca}}
\institute{Department of Electrical and Computer Engineering \\
  University of Waterloo}
\date{\today}


\begin{document}

\begin{frame}
  \titlepage
\end{frame}

\begin{frame}
\frametitle{Software Design Patterns}

\begin{changemargin}{1cm}
Some common re-usable implementation patterns.

Mostly relevant to OOP (but you know OOP).

Many patterns can appear in the same system.

Some conflict, but a large system could have some or all.

\end{changemargin}
\end{frame}

\begin{frame}
\frametitle{Basic Categories}

\begin{changemargin}{1cm}
We will examine three different categories:

\begin{enumerate}
	\item Creation Patterns
	\item Structure Patterns
	\item Behaviour Patterns
\end{enumerate}
\end{changemargin}
\end{frame}

\begin{frame}
\frametitle{Creation Patterns}

\begin{changemargin}{1cm}
Deal with creation of objects. 

Hide creation logic as an implementation detail.

All of the creation logic should be kept together and not spread out over multiple classes.
\end{changemargin}
\end{frame}


\begin{frame}
\frametitle{Creation Patterns: Factory}

\begin{changemargin}{1cm}
Create objects without exposing instantiation logic.

Create an object without specifying the exact class.

Or: use when setup is too complex for the class constructor.

\end{changemargin}
\end{frame}

\begin{frame}
\frametitle{Creation Patterns: Factory}

\begin{changemargin}{1cm}

Use when: creating objects that require a lot of setuconfiguration, or to have flexibility in creating many different kinds of similar classes without having lots of duplicate code.

\end{changemargin}
\end{frame}

\begin{frame}
\frametitle{Creation Patterns: Factory}

\begin{changemargin}{1cm}
Example: Creating a new user. 

The \texttt{User} instantiation requires data not available in the constructor of the \texttt{User} class, such as the user's group.

Solution: \texttt{UserFactory} that creates a new \texttt{User}, populates its data fields as appropriate, and returns the \texttt{User}.

\end{changemargin}
\end{frame}

\begin{frame}[fragile]
\frametitle{Creation Patterns: Factory}
\begin{changemargin}{1cm}
Further examples:

{\scriptsize
\begin{verbatim}
public Rectangle createRectangle() {
  return new RectangleImpl();
}

public FileExample createFileExample() {
  FileExample file = new FileExample();
  file.setOwner(Environment.getInstance().getOwner());
  file.setCreationDate(CalendarDate.today());
}

\end{verbatim}
}
\end{changemargin}
\end{frame}


\begin{frame}
\frametitle{Creation Patterns: Singleton}
\begin{changemargin}{1cm}
Only one instance of a class. Global point of access.

Access controlled through the single point of access.

Use when: only 1 instance of the class may ever exist.

\end{changemargin}
\end{frame}

\begin{frame}
\frametitle{Creation Patterns: Singleton}
\begin{changemargin}{1cm}

Example: program has a \texttt{SecurityManager}.

There is only one instance, accessed by \texttt{SecurityManager.getInstance()}.

To check \texttt{Permissions.READ},\\ \texttt{SecurityManager.getInstance()\\.checkPermission(Permissions.READ)}.


\end{changemargin}
\end{frame}

\begin{frame}[fragile]
\frametitle{Creation Patterns: Singleton}
\begin{changemargin}{1cm}
Code for the Security Manager Example:

\begin{verbatim}
public class SecurityManager() {
  private SecurityManager instance;
  
  private SecurityManager() {/* */}
  
  public SecurityManager getInstance() {
    if (instance == null) {
      instance = new SecurityManager();
    }
    return instance;
  }
}  
\end{verbatim}
\end{changemargin}
\end{frame}


\begin{frame}
\frametitle{Creation Patterns: Template}
\begin{changemargin}{1cm}

An object is used as the basis for making many copies; the copies can then be customized as necessary.

Instead of having many lines of code where data is repeatedly set, define the object once and create copies from that object.

Use when: similar objects will appear often in the system, or to avoid multiple creation routines for the same/similar objects.

\end{changemargin}
\end{frame}

\begin{frame}
\frametitle{Creation Patterns: Template}
\begin{changemargin}{1cm}

Example: every month, the company files tax information with the government. 

Define a template with some fields filled in (e.g., tax ID number, name, address). 

When filing the month's tax information, make a copy of the template, and fill it with the tax transactions for this month.

\end{changemargin}
\end{frame}

\begin{frame}
\frametitle{Creation Patterns: Object Pool}
\begin{changemargin}{1cm}

A set of already-initialized objects are kept on-hand (``in the pool''), ready to use.

Don't allocate and destroy the objects on demand. 

Retrieve from the pool rather than allocate and returned.

When finished it is returned to the pool rather than destroyed. 

If the pool is empty, requesters will simply have to wait.

\end{changemargin}
\end{frame}


\begin{frame}
\frametitle{Creation Patterns: Object Pool}
\begin{changemargin}{1cm}

Limits the amount of concurrent request processing.

Prevents overloading.

Object pools not necessarily permanently-fixed in size.

Use when: allocation and release of resources is expensive, or done very frequently.\\
\quad (even an inexpensive allocation, done a million times...).

\end{changemargin}
\end{frame}

\begin{frame}
\frametitle{Creation Patterns: Object Pool}
\begin{changemargin}{1cm}

Example: pool of workers process student transcript requests. 

Requests are put into a queue. 3 workers (running in separate threads) to take the requests and process them. 

When all workers are currently busy with a request, other students have to wait until a worker becomes free. 
\end{changemargin}
\end{frame}

\begin{frame}
\frametitle{Creation Patterns: Object Pool}
\begin{changemargin}{1cm}


When a particular student's transcript request is at the front of the queue, an available worker takes that request.

The worker fulfills it, returning the transcript information.

The worker returns to the pool, ready to accept a new request.

\end{changemargin}
\end{frame}

\begin{frame}
\frametitle{Structure Patterns}
\begin{changemargin}{1cm}
Structural patterns are used to design the relationships between entities of the system.

\end{changemargin}
\end{frame}


\begin{frame}
\frametitle{Structure Patterns: Get-and-Set}
\begin{changemargin}{1cm}
You should already be familiar with the idea (ECE~150).

Access to variables is moderated through the use of methods that get and set those variables. 

This hides the implementation of the object.

\end{changemargin}
\end{frame}

\begin{frame}
\frametitle{Structure Patterns: Get-and-Set}
\begin{changemargin}{1cm}
Use when: pretty much always.

Example: ... Do I really need to give an example?
\end{changemargin}
\end{frame}


\begin{frame}
\frametitle{Structure Patterns: Adapter}
\begin{changemargin}{1cm}
An adapter converts the interface of one object into another.

The adapter rarely does any work on its own; it mostly converts communication (requests and responses) from one form to another.

Real life analogy: power adapter.

\end{changemargin}
\end{frame}


\begin{frame}
\frametitle{Structure Patterns: Adapter}
\begin{changemargin}{1cm}

When to use: you have two objects or modules which are fundamentally capable of interacting, but have incompatible interfaces.

\end{changemargin}
\end{frame}

\begin{frame}
\frametitle{Structure Patterns: Adapter}
\begin{changemargin}{1cm}

Example: The Linux Software known as WINE. 

It allows (some) Windows programs to work on a Linux system. 

A call to a Windows system function (such as ``draw window'') is handled by WINE.

It converts that request to the equivalent Linux form and send the converted request to the Linux system. 

\end{changemargin}
\end{frame}

\begin{frame}
\frametitle{Structure Patterns: Proxy}
\begin{changemargin}{1cm}

Intermediate object that mediates access to the ``real'' object.

Instead of accessing the target, other classes access the proxy.

The proxy can do one or more things, such as check security or log some data.

\end{changemargin}
\end{frame}

\begin{frame}
\frametitle{Structure Patterns: Proxy}
\begin{changemargin}{1cm}

When to use: when access to a target object should be controlled or monitored.

\end{changemargin}
\end{frame}

\begin{frame}
\frametitle{Structure Patterns: Proxy}
\begin{changemargin}{1cm}

Example: you want to log database queries. 

Where the database access class was used, return the proxy.

Proxy logs the requests and then delegates the execution to the database access class.

\end{changemargin}
\end{frame}

\begin{frame}
\frametitle{Behaviour Patterns}
\begin{changemargin}{1cm}
We use behaviour patterns to model common communications between different objects in the system.

\end{changemargin}
\end{frame}


\begin{frame}
\frametitle{Behaviour Patterns: Iterator}
\begin{changemargin}{1cm}
An Iterator is used to access all of the elements of a collection, such as a List, without knowing what the internal structure is.

The List may be implemented as an array or linked list. 

The Iterator is a common way of accessing the elements (with the \texttt{next()} method in Java) regardless of implementation.

Use iterator on any collection: List, Queue, Stack, etc.

\end{changemargin}
\end{frame}

\begin{frame}
\frametitle{Behaviour Patterns: Iterator}
\begin{changemargin}{1cm}
Use when: examining the elements of a collection, if the collection might be of more than one type.

\end{changemargin}
\end{frame}

\begin{frame}
\frametitle{Behaviour Patterns: Iterator}
\begin{changemargin}{1cm}
Example: Professor wants to check programming assignments.

Rather than have a bunch of special-case code, use the iterator to go through each element in the collection.

\end{changemargin}
\end{frame}

\begin{frame}
\frametitle{Behaviour Patterns: Memento}
\begin{changemargin}{1cm}
Capture a copy of the internal state and provide a means for restoring the object to that state. 

When to use: when we need to save and restore state. 

\end{changemargin}
\end{frame}

\begin{frame}
\frametitle{Behaviour Patterns: Memento}
\begin{changemargin}{1cm}

Example: Editing a document. 

The user can make changes, but if they click on cancel then the object is reverted to the previous state. 

\end{changemargin}
\end{frame}

\begin{frame}
\frametitle{Behaviour Patterns: Visitor}
\begin{changemargin}{1cm}

A way of separating an algorithm from an object. 

Have a new operation on code that you cannot (or choose not to) modify. 

The class with the algorithm ``visits'' (examines) the elements.

\end{changemargin}
\end{frame}

\begin{frame}
\frametitle{Behaviour Patterns: Visitor}
\begin{changemargin}{1cm}

Use when: defining a new operation without changing the elements on which it operates.

\end{changemargin}
\end{frame}

\begin{frame}
\frametitle{Behaviour Patterns: Visitor}
\begin{changemargin}{1cm}

Example: You want to print a collection of elements out in a user-readable format to the screen. 

It is some third party code that you cannot modify.

You can write a visitor that goes to each element and prints it to the screen in the right format. 

\end{changemargin}
\end{frame}

\begin{frame}
\frametitle{Behaviour Patterns: Listener}
\begin{changemargin}{1cm}

Method that executes in response to a change in the system. 

It is called a listener because it ``listens'' for changes in the target object. 

When it ``hears'' such a change, it takes action. The action can be anything.

\end{changemargin}
\end{frame}

\begin{frame}
\frametitle{Behaviour Patterns: Listener}
\begin{changemargin}{1cm}

Register listeners: tell the system you care about changes in something.

The system (or your code) will also have to indicate when there has been a change.

When to use: when an object needs to be notified of a change in another object.

\end{changemargin}
\end{frame}

\begin{frame}
\frametitle{Behaviour Patterns: Listener}
\begin{changemargin}{1cm}

Example: program user interface. 

When the data is changed, the listener is executed, and updates the screen so the change is visible to the user.

\end{changemargin}
\end{frame}

\begin{frame}
\frametitle{Behaviour Patterns: Listener}
\begin{changemargin}{1cm}
In an upcoming lecture, we will talk about listener implementation in more detail.

You may have already started using listeners in your lab.
\end{changemargin}
\end{frame}

\end{document}
