\documentclass[10pt]{article}
\usepackage[dvipsnames]{xcolor}
\usepackage{tikz}
\usepackage{url}
\usepackage{multicol}
\usepackage{xspace}
\usepackage{pstricks}
\usepackage{wrapfig}
\usepackage[section]{placeins}
\usepackage{wrapfig}
\usepackage{amssymb}
\usepackage[default]{droidserif}
\usepackage[T1]{fontenc}

\usepackage{listings}

%\usepackage{algorithm2e}
\usetikzlibrary{arrows,automata,shapes}
\tikzstyle{block} = [rectangle, draw, fill=blue!20, 
    text width=2.5em, text centered, rounded corners, minimum height=2em]
\tikzstyle{bw} = [rectangle, draw, fill=blue!20, 
    text width=4em, text centered, rounded corners, minimum height=2em]

\newcommand{\handout}[5]{
  \noindent
  \begin{center}
  \framebox{
    \vbox{
      \hbox to 5.78in { {\bf ECE155: Engineering Design with Embedded Systems } \hfill #2 }
      \vspace{4mm}
      \hbox to 5.78in { {\Large \hfill #5  \hfill} }
      \vspace{2mm}
      \hbox to 5.78in { {\em #3 \hfill #4} }
    }
  }
  \end{center}
  \vspace*{4mm}
}

\newcommand{\lecture}[4]{\handout{#1}{#2}{#3}{#4}{Lab #1}}
\topmargin 0pt
\advance \topmargin by -\headheight
\advance \topmargin by -\headsep
\textheight 8.9in
\oddsidemargin 0pt
\evensidemargin \oddsidemargin
\marginparwidth 0.5in
\textwidth 6.5in

\parindent 0in
\parskip 1.5ex
%\renewcommand{\baselinestretch}{1.25}

\newcommand{\todo}[1]{{\red\textbf{TODO: }#1}\xspace}

\lstset{language=java, 
        basicstyle=\ttfamily,columns=fullflexible,
%	keywordstyle=\color{Blue},          % keyword style
%	commentstyle=\color{OliveGreen}\textit,       % comment style
%	identifierstyle=\color{Black},
%	stringstyle=\color{BrickRed},
	mathescape=true,
	tabsize=3,
	showstringspaces=false}

\begin{document}

\lecture{1 (Reading Sensors \& The Android API) --- Assessment}{Spring 2014}{Prepared by Kirill Morozov}{version 1.21}

You are responsible for conforming to ``University of Waterloo Policy 71: Student Academic Discipline.''  Students complete Part I of this form.  The TA conducting the demo completes the rest after the demo. We are entering marks based on this form, so if no form exists, you get no marks.

\vspace{-1em}

\section*{Part I: Student Comments}
The design (check one of the following): 
\begin{itemize}
\renewcommand{\labelitemi}{$\Box$}
\item Does not incorporate others' work with the exception of the university-provided materials.
\item Incorporates the work of others as indicated in the notes below. 
\end{itemize}
By signing below, I confirm that we wrote the submitted lab code and that it has not been previously submitted for academic credit at this or any other academic institution, except as noted below.

\begin{tabular*}{1\textwidth}{@{\extracolsep{\fill} }c|c|c|c|}
\hline
&Student Name & UW Student ID \# & Signature \\
\hline
Student 1&&&\\[1em]
\hline
Student 2&&&\\[1em]
\hline
Student 3 &&&\\[1em]
\hline
\end{tabular*}

\paragraph{Notes:}


\newpage
\section*{Part II: Demonstration Checklist}
\paragraph{Software Design Checklist (1 mark for each checklist item satisfied)}
\begin{itemize}
\renewcommand{\labelitemi}{$\Box$}
\item Solution was committed to SVN and compiles without errors \\
(This item is mandatory; you get a 0 for broken or uncommitted solutions.)
\item Output labels obviously correspond to appropriate sensors.
\item Light sensor output is correct.
\item App displays all axes of the magnetic field sensor, along with maximum absolute values reached during a run.
\item App displays all axes of accelerometer are displayed, along with maximum absolute values reached during a run.
\item App displays all axes of the rotation sensor, along with maximum absolute values reached during a run.
\item App uses the LineGraphView (or your own comparable alternative) to display accelerometer values.
\item All data is visible, or user can scroll the screen to access all output data
\item User can reset the maximum recorded values (0 marks, optional)
\renewcommand{\labelitemi}{$\Box\Box$}
\item The design and implementation follow good engineering design. Examples: not over-using global variables, avoiding unnecessary code duplication, and giving variables descriptive names. This checklist item is worth 2 marks.
\end{itemize}

\newpage

\section*{Part III: TA Comments}

\paragraph{Lab \#:}
\paragraph{Group \#:}
\paragraph{Date and Time:}

\paragraph{TA Name:}
\paragraph{Signature:}

\paragraph{Mark:}

\paragraph{Notes:}



\end{document}

